\documentclass{article}

\usepackage[spanish]{babel}
\usepackage[utf8]{inputenc}
\usepackage{paralist}
\usepackage[usenames]{color}
\usepackage{colortbl}
\usepackage{url}
\usepackage{color}
\usepackage{mathtools}
\usepackage[many]{tcolorbox}
\usepackage{xypic}


\usepackage{xspace}
\usepackage{xcolor}
\usepackage{color}
\usepackage{xifthen}        % for conditional commands
\usepackage{authblk}

\newboolean{if-as-case}
\setboolean{if-as-case}{true}



\newcommand{\quo}[1]{``#1''}
\definecolor{myred}{rgb}{0.4,0,0}

\newcommand{\mktype}[1]{\mathsf{\color{myred}#1}}
\newcommand{\tint}{\mktype{int}}
\newcommand{\parens}[1]{(#1)}

\def\colorElem{\color{blue}}


\newcommand{\pcase}[3]{\xcase{myblue}{#1}{#2}{#3}}
\ifthenelse{\boolean{if-as-case}}{
  \newcommand{\pif}[3]{\pcase{#1}{#2}{#3}}
}{
  \newcommand{\pif}[3]{\mkkeyword{if}~#1~\mkkeyword{then}~#2~\mkkeyword{else}~#3}
}
\newcommand{\End}[1][\e]{#1}
\newcommand{\tdone}{{\bullet}}
\newcommand{\tend}{{\circ}}
\newcommand{\tbarrier}[1]{{\talloblong}#1}
\newcommand{\cin}[2][.]{{?}#2#1}
\newcommand{\cout}[2][.]{{!}#2#1}
\newcommand{\bin}[2][\e]{{?}\parens{#2:#1}.}
\newcommand{\bout}[2][\e]{{!}\parens{#2:#1}.}
\newcommand{\tcase}[3][\e]{\xcase{myred}{#1}{#2}{#3}}
\ifthenelse{\boolean{if-as-case}}{
  \newcommand{\tif}[3][\e]{\tcase[#1]{#2}{#3}}
}{
  \newcommand{\tif}[3][\e]{\mkkeyword{if}~#1~\mkkeyword{then}~#2~\mkkeyword{else}~#3}
}
\newcommand{\xpoint}[3]{#1 \mathrel{#2} #3}
\newcommand{\tbranch}[3][p]{\xpoint{#2}{\prescript{}{#1}{\&}}{#3}}
\newcommand{\tchoice}[3][p]{\xpoint{#2}{\prescript{}{#1}{\oplus}}{#3}}

\newcommand{\T}{T}


\title{Inferencia de tipos sesión probabil\'isticos}

\author{
Iv\'an Pondal\\
Director: Hern\'an Melgratti
}

\date{\today}

\begin{document}
\maketitle


\newpage

\section*{Contexto}

El desarrollo de software distribuido, ya sea en sus variantes
tradicionales como as\'i tambi\'en en sus nuevas encarnaciones, tales como
el \emph{Cloud/Fog Computing} o la Internet de las
cosas (IoT por su sigla en ingl\'es), se centran fundamentalmente en
la integraci\'on, cooperaci\'on y comunicaci\'on de componentes,
generalmente existentes y desarrolladas de manera independiente.
%
Mas all\'a del desaf\'io tecnol\'ogico que tal integraci\'on
presupone, el problema principal, desde el punto de vista del
desarrollo, es  razonar sobre las propiedades de tales sistemas en
t\'erminos de las propiedades de sus componentes y de los 
mecanismos  utilizados para su composici\'on.
%
%
Es por ello que los \'ultimos a\~nos testimonian un auge en el
desarrollo de {\em t\'ecnicas de descripci\'on} de interfaces y de
{\em soporte a nivel de lenguajes de programaci\'on para el desarrollo
  de aplicaciones correctas por construcci\'on}.  Los tipos
comportamentales, tales c\'omo los tipos de sesi\'on o {\em type-states}
constituyen un ejemplo paradigm\'atico de esta l\'inea de trabajo. La
idea central consiste en la utilizaci\'on de sistemas de tipo (como
extensi\'on de los utilizados en los lenguajes existentes) para
garantizar est\'aticamente propiedades relacionadas con la
interacci\'on de componentes.
%

Los últimos años testimonian un aumento sustancial en el desarrollo de tipos comportamentales y, en 
especial, los {\em tipos de sesi\'on} se han
consolidado como un formalismo central para el an\'alisis modular de aplicaci\'ones distribuidas
 basadas en procesos que comunican a trav\'es de canales (para una introducci\'on ver~\cite{HuttelEtAl16}).
 Solo a modo ilustrativo de la actividad en este tema, se listan a continuaci\'on art\'iculos publicados en 
el \'ultimo a\~no en las mayores conferencias del \'area~\cite{DBLP:journals/pacmpl/GhilezanPPSY21,
DBLP:journals/pacmpl/HinrichsenBK20,
DBLP:journals/pacmpl/Castro-PerezY20,
DBLP:conf/ecoop/ImaiNYY19,
DBLP:conf/ecoop/000119,
DBLP:conf/esop/JongmansY20,
DBLP:conf/esop/VasconcelosCAM20,
DBLP:conf/concur/Horne20,
DBLP:conf/concur/DasP20,
DBHPS21,DBLP:conf/cc/Miu0Y021, DBLP:journals/pacmpl/GriesemerHKLTTW20,
DBLP:journals/pacmpl/MajumdarYZ20,
DBLP:conf/concur/InversoMPTT20}.

Los tipos de sesi\'on fueron originalmente propuestos en~\cite {Honda93}, y promulgan la idea de estructurar 
la comunicaci\'on entre componentes alrededor del concepto de {\em sesi\'on}. Una {sesi\'on}
es un canal privado que permite conectar a dos (a veces m\'as)
procesos, donde cada uno posee un \emph{endpoint} de la sesi\'on y lo debe usar siguiendo 
un protocolo bien  especificado, llamado \emph{tipo de
  sesi\'on}, que restringe el secuencia de mensajes que se pueden
enviar y recibir a trav\'es de ese endpoint. Como ejemplo, el tipo de
sesi\'on
\begin {equation}
  \label {eq:bare.auction}
  \cout \tint \parens {
    \tbranch [] \tend {
      \cin \tint \parens {
        \tchoice [] \tend \T
      }
    }
  }
\end {equation}
podr\'ia describir (parte de) un protocolo que debe seguir un proceso comprador para 
participar en una subasta: el proceso debe env\'iar un valor entero que es su oferta
($ \cout [] \tint $) y espera una decisi\'on del subastador. El
protocolo puede proceder en este punto en dos formas distintas correspondientes a las
dos ramas del operador $ \tbranch [] {} {}$. El subastador puede declarar que el art\'iculo se vende, en cuyo
caso la sesi\'on termina inmediatamente ($ \tend $), o puede contraofertar un nuevo valor ($ \cin [] \tint
$). En ese momento, el comprador puede elegir ($ \tchoice [] {} {} $)
terminar la subasta o reiniciar el protocolo con otra oferta, aqu\'i
denotado por $\T$. Dependiendo de si el mecanismo de tipado es est\'atico o din\'amico, el 
 tipo sesi\'on se utiliza  durante la  compilaci\'on o la ejecuci\'on 
 para asegurar que un programa utiliza el canal de sesi\'on de acuerdo a su tipo. 
Como en cualquier sistema de tipos, un programa bien tipado goza de las propiedades que 
garantiza el sistema propuesto. En general, se asegura la propiedad de  {\em type-safety} 
(es decir los procesos intercambian informaci\'on que tiene el tipo esperado), la adherencia 
al protocolo (las secuencias de acciones sobre un canal siguen el orden descripto por el protocolo), pero tambi\'en pueden garantizar propiedades  como 
la  ausencia de {\em deadlocks} y {\em livelocks}~\cite{HuttelEtAl16}.




En~\cite{DBLP:conf/concur/InversoMPTT20} se propuso el uso de tipos de sesi\'on para
razonar probabil\'isticamente sobre propiedades de alcanzabilidad, donde se di\'o
 una interpretaci\'on
\emph{probabil\'istica} a los operadores de selecci\'on. M\'as específicamente, se pasa 
 de una interpretación no determin\'istica a una probabilística y se 
 estudia un sistema de tipos que permita determinar la probabilidad con la que una sesi\'onn en particular
termina \emph{exitosamente}. Dado que no existe una interpretaci\'on universal
 de `` terminación exitosa '', se diferencia la terminaci\'on exitosa de la 
 infructuosa  por medio de un
  {constructor dedicado}.
%
Por ejemplo, en nuestro sistema de tipos podemos refinar
\eqref{eq:bare.auction} as
\begin{equation}
  \label{eq:refined.auction}
  \cout\tint\parens{
    \tbranch[p]\tdone{
      \cin\tint\parens{
        \tchoice[q]\tend\T
      }
    }
  }
\end{equation}
donde el tipo de sesión $\tdone$ indica una terminación exitosa y donde las
ramas  se anotan con probabilidades $p$ y
$q$. En particular, el subastador declara que el artículo se vende con
probabilidad $p$ y  responde con una contraoferta con probabilidad
$1-p$, mientras que el comprador decide abandonar la subasta con
probabilidad $q$ o volver a ofertar  con probabilidad $1-q$.

Si bien la l\'inea trabajo sobre tipos sesi\'on tiene ra\'ices en resultados te\'oricos
sobre modelos fundacionales de la c\'omputaci\'on concurrente, como
los son CCS y el c\'alculo $\pi$, en la actualidad existen extensiones
concretas de sistemas de tipos de lenguajes de programaci\'on de
escala industrial, tales como Scala~\cite{DBLP:conf/pldi/ScalasYB19}, Dotty~\cite{DBLP:conf/pldi/ScalasYB19}, Go~\cite{DBLP:conf/icse/LangeNTY18,DBLP:conf/icse/LangeNTY18}, Rust~\cite{DBLP:journals/corr/abs-1909-05970,DBLP:conf/coordination/LagaillardieNY20}, Haskell~\cite{orchard2017session,DBLP:conf/haskell/LindleyM16}, 
Ocaml~\cite{DBLP:journals/jfp/Padovani17,DBLP:conf/coordination/LagaillardieNY20,DBLP:conf/ecoop/ImaiNYY19}, Erlang~\cite{fowler2016erlang} y han sido aplicados, entre otros, 
al estudio de  {\em smart contracts} y tecnolog\'ias {\em blockchain}~\cite{10.1145/3417516,DBLP:journals/corr/abs-1902-06056}. 

Este no es el caso para los tipos sesi\'on probabilisticos, dado que
 la propuesta en~\cite{DBLP:conf/concur/InversoMPTT20} presenta un sistema de chequeo 
de tipos, pero deja abierta la definici\'on de un algoritmo de inferencia y su implementaci\'on 
en un lenguaje de programaci\'on.

\section*{Objetivo}
A partir de una implementaci\'on existente de tipos sesi\'on, dar una implementaci\'on 
del sistema de tipos probabil\'isticos presentado en~\cite{DBLP:conf/concur/InversoMPTT20}.
En particular, esperamos obtener una versi\'on probabilistica de la  implementaci\'on en Ocaml, denominada FuSe~\cite{DBLP:journals/jfp/Padovani17}.



\section*{Actividades}
\begin{enumerate}
\item  Estudiar y analizar la implementaci\'on de sesiones en FuSe~\cite{DBLP:journals/jfp/Padovani17}, en particular, la codificaci\'on de los tipos sesi\'on en t\'erminos de tuplas.
\item  Redise\~nar  las primitivas de comunicaci\'on brindadas por FuSe para incorporar elecciones 
probabil\'isticas. El paso inicial ser\'a considerar  las primitivas para diferenciar finalizaci\'on exitosa de aquella fallida y proporcionar una primitiva para elecci\'on probabil\'istica. Se considerar\'an principalmente  elecciones binarias (evitando el uso de elecciones generalizadas en t\'ermino de variantes polim\'orficas). 
\item Elaborar  una estrategia para codificar a nivel de tipos las distintas elecciones probabil\'isticas y su combinaci\'on. Se espera poder utilizar variables de tipo frescas asociadas con cada selecci\'on, que 
ser\'an combinadas reutilizando el mecanismo de inferencia est\'andar de Ocaml. El objetivo principal, es codificar la informaci\'on sobre las elecciones probabil\'isticas de manera tal de usufructuar el mecanismo de inferencia de tipos del lenguaje sin necesidad de cambios.
\item Utilizar los tipos inferidos por el compilador consistentes en tipos de sesi\'on anotados con elecciones probabil\'isticas para calcular la probabilidad de terminaci\'on exitosa de una sesi\'on. Para ello se deber\'a computar la probabilidad de absorci\'on de la cadena de Markov asociada al tipo de sesi\'on inferido.
\item Implementaci\'on de un set b\'asico de ejemplos (como los presentados en \cite{DBLP:journals/jfp/Padovani17}) para ilustrar la aplicabilidad de la propuesta. 
\end{enumerate}

{\bf Fecha acordada de finalizaci\'on:  Diciembre 2021}
\bibliographystyle{alpha}
\bibliography{hernan}


\end{document}
