Our implementation stems from the work done in \FuSe. In this section we present
the changes needed on the encoding and types in order to model probabilistic
choices and have the information required to compute the successful termination
of a session type.

\section{Encoding}

\FuSe takes the encoding proposed in \cite{Dardha} further refined
in~\cite{DBLP:journals/jfp/Padovani17}. The main idea is that a sequence of
messages over a session can be modelled as a sequence of communications over
single-use channels. In this model, each message transports a value along a
fresh channel to continue the communication.

This encoding depends on the following types:

\begin{itemize}
	\item $\tbottom$ represents an absent value.
	\item $\tsession{\renc{}}{\senc{}}$ describes channels which can
		receive messages of type $\renc{}$ and send of type $\senc{}$. Both
		$\renc{}$ and $\senc{}$ can be set to $\tbottom$ if the channel does not
		receive or send a value.
\end{itemize}

From this las representation, the following instances ar used to describe the
possible terminations (one difference from \FuSe is the usage of
$\DoneBottomType$ to distinguish each of these).

\begin{itemize}
	\item $\tsession{\DoneBottomType}{\DoneBottomType}$ represents a successful
	termination.
	\item $\tsession{\tbottom}{\tbottom}$ represents an unsuccessful
	termination.
\end{itemize}

The relationship between session types $\SessionType$ and the types
$\tsession\TypeT\TypeS$ is given by the following $\encfun\cdot$ function

\begin{center}
$\displaystyle
  \begin{array}{@{}c@{}}
    \textbf{Probabilistic session types encoding} \\
    \hline
    \hline
    \begin{array}[t]{@{}r@{~}c@{~}l@{}}
      \encfun\Done
      & = &
      \tsession\DoneBottomType\DoneBottomType
      \\
      \encfun\Idle
      & = &
      \tsession\tbottom\tbottom
      \\
      \encfun{\In\Type\SessionType}
      & = &
      \tsession{\encfun\Type\tmul\encfun\SessionType}\tbottom
      \\
      \encfun{\Out\Type\SessionType}
      & = &
      \tsession\tbottom{\encfun\Type\tmul\encfun{\dual\SessionType}}
      \\
      \encfun{\BinaryPBranch[p]{\SessionTypeT}{\SessionTypeS}}
      & = &
      \tsession{\BinaryLabels{\encfun{\SessionTypeT}}{\encfun{\SessionTypeS}}\tmul p}\tbottom
      \\
      \encfun{\BinaryPChoice[p]{\SessionTypeT}{\SessionTypeS}}
      & = &
      \tsession{\tbottom}{\BinaryLabels{\encfun{\dual\SessionTypeT}}{\encfun{\dual\SessionTypeS}}\tmul p}
      \\
      \encfun\etvar
      & = &
      \tsession{\renc\etvar}{\senc\etvar}
      \\
      \encfun{\sdual\etvar}
      & = &
      \tsession{\senc\etvar}{\renc\etvar}
    \end{array}
  \end{array}
$
\end{center}
