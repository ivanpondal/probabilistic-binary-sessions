Our implementation stems from the work done in \FuSe. In this section we present
the changes needed on the encoding and types in order to model probabilistic
choices and have the information required to compute the successful termination
of a session type.

\section{Encoding}

\FuSe takes the encoding proposed in \cite{Dardha} further refined
in~\cite{DBLP:journals/jfp/Padovani17}. The main idea is that a sequence of
messages over a session can be modelled as a sequence of communications over
single-use channels. In this model, each message transports a value along a
fresh channel to continue the communication.

The encoding uses the following types:

\begin{itemize}
	\item $\tbottom$ represents an absent value.
	\item $\tsession{\renc{}}{\senc{}}$ describes channels which can
		receive messages of type $\renc{}$ and send of type $\senc{}$. Both
		$\renc{}$ and $\senc{}$ can be set to $\tbottom$ if the channel does not
		receive or send a value.
\end{itemize}

From this last representation, the following instances are used to describe
termination states (one difference from \FuSe is the use of $\DoneBottomType$ to
model the unsuccessful termination state).

\begin{itemize}
	\item $\tsession{\DoneBottomType}{\DoneBottomType}$ represents a successful
	termination.
	\item $\tsession{\tbottom}{\tbottom}$ represents an unsuccessful
	termination.
\end{itemize}

The relationship between session types $\SessionType$ and the types
$\tsession\TypeT\TypeS$ is given by the following $\encfun\cdot$ function

\begin{center}
$\displaystyle
  \begin{array}{@{}c@{}}
    \textbf{Probabilistic session types encoding} \\
    \hline
    \hline
    \begin{array}[t]{@{}r@{~}c@{~}l@{}}
      \encfun\Done
      & = &
      \tsession\DoneBottomType\DoneBottomType
      \\
      \encfun\Idle
      & = &
      \tsession\tbottom\tbottom
      \\
      \encfun{\In\Type\SessionType}
      & = &
      \tsession{\encfun\Type\tmul\encfun\SessionType}\tbottom
      \\
      \encfun{\Out\Type\SessionType}
      & = &
      \tsession\tbottom{\encfun\Type\tmul\encfun{\dual\SessionType}}
      \\
      \encfun{\BinaryPBranch[p]{\SessionTypeT}{\SessionTypeS}}
      & = &
      \tsession{\BinaryLabels{\encfun{\SessionTypeT}}{\encfun{\SessionTypeS}}\tmul p}\tbottom
      \\
      \encfun{\BinaryPChoice[p]{\SessionTypeT}{\SessionTypeS}}
      & = &
      \tsession{\tbottom}{\BinaryLabels{\encfun{\dual\SessionTypeT}}{\encfun{\dual\SessionTypeS}}\tmul p}
      \\
      \encfun\etvar
      & = &
      \tsession{\renc\etvar}{\senc\etvar}
      \\
      \encfun{\sdual\etvar}
      & = &
      \tsession{\senc\etvar}{\renc\etvar}
    \end{array}
  \end{array}
$
\end{center}

homomorphically extended to missing types. We assume that for each session type
variable $\etvar$ there are two distinct type variables $\senc\etvar$ and
$\renc\etvar$.

As an example, the session type $\In\tvar\etvar$ is coded as $\tsession{\tvar
\tmul \tsession{\renc\etvar}{\senc\etvar}}\tbottom$, describing a channel which
receives a message of type $\tvar \tmul \tsession{\renc\etvar}{\senc\etvar}$; a
component of type $\tvar$ (the value to communicate) and another of type
$\tsession{\renc\etvar}{\senc\etvar}$ (the channel where the communication will
continue).

When encoding session types that send a value or a choice, we must apply the
dual to the continuation type. This is required because the continuation type
describes the behavior the \emph{receiver} should follow. Additionally, this
implementation detail allows a representation for duality of session types which
works even when a type isn't fully instantiated.

As an example, will take the probabilistic version of the type $\SessionType$
encoding given at~\cite{DBLP:journals/jfp/Padovani17} $\SessionType =
\BinaryPChoice[p]{\Tag[\Out\tvarA\In\tvarB\Done]}{\Tag[\Idle]}$
\[
\begin{array}{@{}rcl@{}}
  \encfun\SessionType
  & = & \tsession\tbottom{
	  \BinaryLabels{\encfun{\In\tvarA\Out\tvarB\Done}}{\encfun\Idle}\tmul p
    }
  \\
  & = & \tsession\tbottom{
	  \BinaryLabels{\tsession{\tvarA\tmul\encfun{\Out\tvarB\Done}}{\tbottom}}
	               {\tsession\tbottom\tbottom}\tmul p
  }
  \\
  & = & \tsession\tbottom{
	  \BinaryLabels{\tsession{\tvarA\tmul\tsession{\tbottom}
	                                              {\tvarB\tmul\encfun{\Done}}}
	                         {\tbottom}}
	               {\tsession\tbottom\tbottom}\tmul p
  }\\
  & = & \tsession\tbottom{
	  \BinaryLabels{\tsession{\tvarA\tmul\tsession{\tbottom}
						      {\tvarB\tmul\tsession{\DoneBottomType}
	                                                                   {\DoneBottomType}}}
	                         {\tbottom}}
	               {\tsession\tbottom\tbottom}\tmul p
  }
\end{array}
\]

Then, its dual
$\dual\SessionType = \BinaryPBranch{\In\tvarA\Out\tvarB\End}{\Idle}$
\[
\begin{array}{@{}rcl@{}}
  \encfun{\dual\SessionType}
  & = & \tsession{
	  \BinaryLabels{\encfun{\In\tvarA\Out\tvarB\Done}}{\encfun\Idle}\tmul p
    }\tbottom
  \\
  & = & \tsession{
	  \BinaryLabels{\tsession{\tvarA\tmul\encfun{\Out\tvarB\Done}}{\tbottom}}
	               {\tsession\tbottom\tbottom}\tmul p
  }\tbottom
  \\
  & = & \tsession{
	  \BinaryLabels{\tsession{\tvarA\tmul\tsession{\tbottom}
	                                              {\tvarB\tmul\encfun{\Done}}}
	                         {\tbottom}}
	               {\tsession\tbottom\tbottom}\tmul p
  }\tbottom
  \\
  & = & \tsession{
	  \BinaryLabels{\tsession{\tvarA\tmul\tsession{\tbottom}
						      {\tvarB\tmul\tsession{\DoneBottomType}
	                                                                   {\DoneBottomType}}}
	                         {\tbottom}}
	               {\tsession\tbottom\tbottom}\tmul p
  }\tbottom
\end{array}
\]

With this last example we note the encoding for $\dual\SessionType$ can be
obtained from $\SessionType$ by swapping the components of its encoded pair.
Just as with session types, we can prove the following property for
probabilistic types:

\begin{theorem}
  \label{thm:duality}
  If $\encfun\SessionType = \tsession\TypeT\TypeS$, then
  $\encfun{\dual\SessionType} = \tsession\TypeS\TypeT$.
\end{theorem}
\begin{proof}
Follows mutatis mutandis the original proof given at~\cite{Dardha}.
\end{proof}

Equivalently, we may say that
$\encfun\SessionTypeT = \tsession{\Type_1}{\Type_2}$ and
$\encfun\SessionTypeS = \tsession{\TypeS_1}{\TypeS_2}$, therefore
\[
\SessionTypeT = \dual\SessionTypeS
\iff
\encfun\SessionTypeT = \encfun{\dual\SessionTypeS}
\iff
\TypeT_1 = \TypeS_2
\wedge
\TypeT_2 = \TypeS_1
\]

Just as with \FuSe, this allows computing the dual of a session type through
type equivalence. This also holds for unknown session types or partially
instantiated ones: $\encfun\etvar = \tsession{\renc\etvar}{\senc\etvar}$ and
$\encfun{\sdual\etvar} = \tsession{\senc\etvar}{\renc\etvar}$.

\begin{table}[htb]
	\begin{OCamlD}[frame=single]
  type _0
  type _1
  type $p_0$ (* zero probability encoding alias *)
  type $p_1$ (* probability one encoding alias *)
  type ('r,'s) pst (* OCaml syntax for $\tsession\tvarR\tvarS$ *)
  val create  : unit -> ('r,'s) pst * ('s,'r) pst
  val close   : (_0,_0) pst -> unit
  val idle    : (_1,_1) pst -> unit
  val send    : 'a -> (_0,('a * ('s,'r) pst)) pst -> ('r,'s) pst
  val receive : (('a * ('r,'s) pst),_0) pst -> 'a * ('r,'s) pst
  val select_true  : (_0,[`True of ('r,'s) pst |
                         `False of ('c,'d) pst] * $p_1$)
                      -> ('r,'s) pst
  val select_false : (_0,[`True of ('r,'s) pst |
                         `False of ('c,'d) pst] * $p_0$)
                      -> ('c,'d) pst
  val pick   : $p$ ->
             ((_0,[`True of ('r,'s) pst | `False of ('c,'d) pst]
                  * $p_0$) -> 'a) ->
             ((_0,[`True of ('r,'s) pst | `False of ('c,'d) pst]
                  * $p_1$) -> 'a) ->
             (_0,[`True of ('r,'s) pst | `False of ('c,'d) pst]
                 * $p$) -> 'a
  val branch : ([`True of ('r,'s) pst |
                 `False of ('c,'d) pst] * $p$,_0)
                -> [> `True of ('r,'s) pst | `False of ('c,'d) pst]
	\end{OCamlD}
	\caption{Probabilistic session types \OCaml interface.}
	\label{tab:signature}
\end{table}

Having chosen a representation for probabilistic session types, we show the
implementation of the \OCaml interface in Table \ref{tab:signature}. There is a
direct correspondence between the signature of the functions present at Table
\ref{tab:signature} and the primitives introduced at Table \ref{tab:prob_api}.
It's easy to see how this encoding results in session types which are hard to
read. This problem gets worse once the protocols get more complex. For this
reason, we have extended \texttt{rosetta}, an auxiliary tool from \FuSe, which
implements the inverse function to the encoding and prints the types using the
original probabilistic types notation (Section \ref{sec:gramatica_prob}). When
presenting session types inferred by \OCaml we'll use the format generated by
the tool in order to ease the reading (Details on \texttt{rosetta} are given on
Chapter \ref{cap:prob_exito}).

\subsection{Probability encoding}

In our session types, choices carry in their encoding an associated probability
$p$ which is also present as an argument of the \OI{pick} primitive. We begin by
defining $p$ as the probability of sending $\BinaryPChoice{\etvar}{\etvarB}$ or
receiving $\BinaryPBranch{\etvar}{\etvarB}$ the choice to continue with the
$\Tag[True]$ branch of a session type.

Normally we would consideru a floating point representation like \tfloat for
this scenario, however this would not provide any compilation time information
on the value of $p$. Having the value as a type allows for stronger interfaces
as is the case with \OI{pick}, where the probability $p$ received as an argument
must match with the one of the session type over which it's being applied (Table
\ref{tab:prob_api}).

Another advantage is being able to use this value for any sort of static
analysis, a hard requirement for computing the success probability of a session
type.

\subsection{Modeling natural and rational numbers}

For this work we chose to model $p$ as a rational number which is present at
type level. To do so, we started by writing a type definition for natural
numbers based on the construction through a successor function.

\begin{table}[htb]
\begin{OCamlD}[frame=single]
      type _z
      type _s
      type _ nat = Z : _z nat | S : 'a nat -> (_s * 'a) nat
\end{OCamlD}
\caption{Natural numbers and zero representation}
\label{tab:def_nat}
\end{table}

First we define the types \OI{_z} and \OI{_s} that will be used as labels to
distinguish at type level between zero and the successor function. Then we
define \OI{nat} as a GADT (Generalized Algebraic Data Type) present in
\OCaml~\cite{YallopM} which allows adding restrictions on the parameters of its
constructors. We define a constructor \OI{Z} which acts as zero and \OI{S} for
the successor function, adding one to the natural it receives as an argument.
Finally, naturals can be built trivially as shown in Table \ref{tab:ex_nat}.

\begin{table}[htb]
\begin{OCamlD}[frame=single]
      let zero : _z nat = Z
      let one : (_s * _z) nat = S Z
      let two : (_s * (_s * _z)) nat = S (S Z)
\end{OCamlD}

\caption{Construction of natural numbers}
\label{tab:ex_nat}
\end{table}

Once we have a representation for natural numbers, we can define \OI{frac} as
another GADT that allows building rational numbers. In this case we use a pair
where the first component is the numerator and the second the denominator,
disallowing a zero division through a type restriction. This way, rationals can
be defined as follows:

\begin{table}[htb]
\begin{OCamlD}[frame=single]
	type _ frac = Fraction : 'a nat * (_s * 'b) nat ->
	                        ('a nat * (_s * 'b) nat) frac
\end{OCamlD}
\caption{Rational numbers representation}
\label{tab:def_rational}
\end{table}

As an example we show the construction of $\frac{1}{2}$.

\begin{table}[H]
\begin{OCamlD}[frame=single]
	let one_half : ((_s * _z) nat *
	                (_s * (_s * _z)) nat) frac
	                = Fraction (S Z, S (S Z))
\end{OCamlD}
\caption{Rational number example}
\label{tab:ex_rational}
\end{table}
