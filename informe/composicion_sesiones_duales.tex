Nos resta introducir un tipo más a nuestra extensión, la composición de tipos sesión duales representada como $\ClosedSessionType$. La misma encapsula la
unión de los endpoints de tipo $\T$ y $\dual\T$ con probabilidad de éxito $p =
\encfun\T = \encfun{\dual\T}$. \todo{Agregar breve descripción con regla de tipado T-Par}

Este tipo se presenta al momento de crear una sesión y su función es puramente
informativa, guarda la probabilidad de éxito de la sesión. En nuestra extensión
queremos capturar esta información ya que representa uno de los objetivos del
cálculo probabilístico, obtener la probabilidad de éxito de una sesión mediante
la información de tipo de sus primitivas.

Elegir una representación adecuada para este tipo presenta algunos desafíos.
La misma debe capturar la información de tipo de sus endpoints ($\T$ o su
dual $\dual\T$) para permitir el cálculo de éxito y a su vez está sujeta a la
combinación probabilística vista en Definición \ref{def:ccomb}.

\begin{table}[htb]
	\begin{OCamlD}[frame=single]
  type ('r,'s) cpst (* sintáxis en OCaml para sesión cerrada
                     con endpoint de tipo $\tsession\tvarR\tvarS$ *)
  val create  : ?st:('r, 's) cpst -> unit -> ('r,'s) pst * ('s,'r) pst
  val cst_placeholder : ('a, 'b) cpst
	\end{OCamlD}
	\caption{Interfaz \OCaml para tipos sesión probabilísticos.}
	\label{tab:create_cpst_sig}
\end{table}

Nuestra representación toma la forma de un argumento opcional en la primitiva
\OI{create}. Dado que la función de este tipo es informativa, el uso de un
argumento opcional permite ignorarlo y aún así disponer de las garantías de
tipo que provee nuestra extensión. Podemos observar que el tipo unifica con uno
de los endpoints de la sesión, $\encfun\T = \tsession\tvarR\tvarS$. También
podría haber tomado su dual, $\encfun{\dual\T} = \tsession\tvarS\tvarR$, ya que
el propósito es disponer del tipo de la sesión completa para calcular su
probabilidad de éxito.

Dado que sólo nos interesa el tipo final de nuestro argumento opcional, el
valor a utilizar es indistinto; para ello tenemos la constante
\OI{cst_placeholder} que tipa como una sesión cerrada pero su implementación es
un valor de tipo \OI{unit}.

\section{Ejemplos de uso}

Veamos qué ocurre al obtener la composición de tipos para los ejemplos vistos en la introducción:

\EchoClientClosedSession

En este programa, primero creamos los endpoints utilizando \OI{st} como
argumento para obtener así la información de tipo de la sesión. Luego cada
endpdoint es consumido acorde al comportamiento de la sesión, permitiendo así
inferir su tipo. Habiendo unificado nuestro argumento opcional \OI{st} con el
tipo de la sesión, nuestra herramienta traductora \texttt{rosetta} es capaz de
calcular la probabilidad de éxito de la misma. Para este ejemplo, nuestro programa
\OI{run_echo_client_example} llevaría el siguiente tipo:

\begin{table}[htb]
	\begin{OCamlD}
    val run_echo_client_example : ?st:$\ClosedSessionType[1]$ -> unit -> int
	\end{OCamlD}
\end{table}

La probabilidad de éxito es $1$ lo cual no sorprende ya que este ejemplo es
completamente determinístico. Contemplemos ahora un escenario con algún tipo de
elección probabilística:

\CoinFlipEchoClientClosedSession

El único cambio con respecto al ejemplo anterior es el uso de
\OI{coin_flip_echo_client} que envía con probabilidad $\frac{1}{2}$, caso
contrario finaliza con \OI{idle}.

\begin{table}[htb]
	\begin{OCamlD}
 val run_coin_flip_echo_client_example : ?st:$\ClosedSessionType[0.5]$ -> unit -> int
	\end{OCamlD}
\end{table}

En este caso, la probabilidad de éxito es de $\frac{1}{2}$ lo cual coincide con
el comportamiento de su sesión.

\section{Interacción con sesiones multi-sesión}

Si recordamos la definición del combinador probabilístico, esta incluía la
combinación de sesiones cerradas. En particular tenemos que
$\csum{\ClosedSessionType[q]}{\ClosedSessionType[r]} =
\ClosedSessionType[pq+(1-p)r]$. A diferencia de la combinación de endpoint
donde el tipo combinado lleva información del comportamiento de la sesión, acá
se trata nuevamente de un dato informativo; la probabilidad de éxito de una
sesión que puede tomar distinta forma según una elección externa. 

Veamos un ejemplo de tal combinación:

\PickIdleCloseRunEchoClient

En este caso, \OI{ep} tiene tipo $\BinaryPChoice[p]{\Done}{\Idle}$; esta
elección afecta el tipo de \OI{st} ya que vemos está presente en ambas ramas.
La sesión cerrada \OI{st} tiene tipo $\ClosedSessionType[1]$ en cada rama (se
trata de una sesión sin elecciones probabilísticas), la combinación no afecta
su tipo final.

Para el siguiente ejemplo, \OI{st} captura la probabilidad de éxito de dos
sesiones distintas en cada rama:

\InvalidPickIdleCloseRunEchoClientOrCoinFlip

En este caso podemos observar se obtiene un error de tipado ya que al escoger
$\Tag[True]$, \OI{st} toma el tipo $\ClosedSessionType[0,5]$ mientras que en
la rama $\Tag[False]$ este es de tipo $\ClosedSessionType[1]$.

Dado que se trata de un tipo informativo, cabe resaltar este ejemplo tipa si
quitamos el argumento opcional:

\NoOptionalValidPickIdleCloseRunEchoClientOrCoinFlip

El error surge al buscar la aplicación del combinador probabilístico sobre los
tipos de las sesiones finalizadas en la firma de nuestro programa.

Nos encontramos con la misma limitación del \OI{pick} solo que aplicado al tipo
de sesiones concluidas. Siguiendo la misma motivación que con \OI{pick_2ch},
decidimos agregar una primitiva que capture en su tipo esta nueva combinación.
Presentamos a continuación la primitiva \OI{pick_2st} y su tipo en Tabla
\ref{tab:prob_api_pick_2st}. Así como \OI{pick_2ch} permite combinar las
elecciones de exactamente dos sesiones, aquí podemos combinar exactamente un
endpoint y una sesión cerrada.

\begin{table}[htb]
\begin{OCamlD}[frame=single]
	val pick_2st:
	    $p$ -> $(\BinaryPChoice[0]{\etvarA}{\etvarB}$ -> $\ClosedSessionType[r]$ -> $\alpha)$
	      -> $(\BinaryPChoice[1]{\etvarA}{\etvarB}$ -> $\ClosedSessionType[q]$ -> $\alpha)$
	      -> $\BinaryPChoice[p]{\etvarA}{\etvarB}$ -> $\ClosedSessionType[pq + (1 - p)r]$ -> $\alpha$
\end{OCamlD}
\caption{Interfaz de primitiva \OI{pick_2st}}
\label{tab:prob_api_pick_2st}
\end{table}

El uso de la primitiva es análogo al de \OI{pick_2ch} con la diferencia de que
el segundo argumento de la función de cada rama así como el tipo combinado
final es sobre sesiones combinadas, no elecciones.

\ValidPickIdleCloseRunEchoClientOrCoinFlip

De esta manera, \OI{st} tipa correctamente en cada rama y su información de
tipo permite el cálculo de su combinación ponderada:

\begin{table}[htb]
	\begin{OCamlD}
  val pick_idle_close_and_run_echo_client_or_coin_flip :
              ?st:$\ClosedSessionType[0.75]$ -> unit -> $\BinaryPChoice[\frac{1}{2}]{\Done}{\Idle}$ -> int
	\end{OCamlD}
\end{table}
