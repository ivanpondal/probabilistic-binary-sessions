\chapter{Tipos sesión}
\section{Introducción}

Los tipos sesión [TODO: citas] se han consolidado como un formalismo para el
análisis modular de comunicación entre procesos. Una \emph{sesión} es un canal
privado que conecta dos procesos, cada uno con su conector u \emph{endpoint} que
establece qué estructura tiene la comunicación mediante una especificación; el
\emph{tipo sesión}. A modo de ejemplo, el tipo sesión

\begin{equation}
    \label{eq:bare.auction}
    \cout \tint \parens {
        \tbranch [] \tend {
            \cin \tint \parens {
                \tchoice [] \tend \T
            }
        }
    }
\end{equation}

podría describir (parte de) un protocolo que debe seguir un proceso comprador
para participar en una subasta: el proceso debe enviar un valor entero que es su
oferta ($ \cout [] \tint $) y espera una decisión del subastador. El protocolo
puede proceder en este punto en dos formas distintas correspondientes a las dos
ramas del operador $ \tbranch [] {} {}$. El subastador puede declarar que el
artículo se vende, en cuyo caso la sesión termina inmediatamente ($ \tend $), o
puede contraofertar un nuevo valor ($ \cin [] \tint $). En ese momento, el
comprador puede elegir ($ \tchoice [] {} {} $) terminar la subasta o reiniciar
el protocolo con otra oferta, aquí denotado por $\T$.

En~\cite{DBLP:conf/concur/InversoMPTT20} se propuso el uso de tipos de sesión
para razonar probabilísticamente sobre propiedades de alcanzabilidad, donde se
dio una interpretación \emph{probabilística} a los operadores de selección. Más
específicamente, se pasa de una interpretación no determinística a una
probabilística y se estudia un sistema de tipos que permita determinar la
probabilidad con la que una sesión en particular termina \emph{exitosamente}.
Dado que no existe una interpretación universal de ``terminación exitosa'', se
diferencia la terminación exitosa de la infructuosa por medio de un constructor
dedicado.

Por ejemplo, podemos refinar \eqref{eq:bare.auction} como
\begin{equation}
    \label{eq:refined.auction}
    \cout\tint\parens{
        \tbranch[p]\tdone{
            \cin\tint\parens{
                \tchoice[q]\tend\T
            }
        }
    }
\end{equation}

donde el tipo sesión $\tdone$ indica una terminación exitosa y donde las
ramas  se anotan con probabilidades $p$ y $q$. En particular, el subastador
declara que el artículo se vende con probabilidad $p$ y  responde con una
contraoferta con probabilidad $1-p$, mientras que el comprador decide abandonar
la subasta con probabilidad $q$ o volver a ofertar  con probabilidad $1-q$.