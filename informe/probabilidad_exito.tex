En secciones anteriores hablamos de la probabilidad de éxito pero aún no dimos
una definición formal para la misma. Intuitivamente, la probablidad se computa
considerando todos los caminos en la estructura de $\SessionType$ que llevan a
un estado $\Done$. Formalmente:

\begin{definition}[Probabilidad de éxito]
  \label{def:pr}
	La \emph{probabilidad de éxito} de un tipo sesión $\SessionType$,
	denotada como $\psfun\SessionType$, se encuentra definida por
	las siguientes ecuaciones:
  \[
    \begin{array}{r@{~}c@{~}l}
      \psfun\Idle & = & 0 \\
      \psfun\Done & = & 1 \\
    \end{array}
    \quad
    \begin{array}{r@{~}c@{~}l}
      \psfun{\In\Type\SessionType} & = & \psfun\SessionType \\
      \psfun{\Out\Type\SessionType} & = & \psfun\SessionType \\
    \end{array}
    \quad
    \begin{array}{r@{~}c@{~}l}
      \psfun{\BinaryPBranch\SessionTypeT\SessionTypeS} & = & p\psfun\SessionTypeT + (1-p)\psfun\SessionTypeS \\
      \psfun{\BinaryPChoice\SessionTypeT\SessionTypeS} & = & p\psfun\SessionTypeT + (1-p)\psfun\SessionTypeS \\
    \end{array}
  \]
\end{definition}

Para un tipo sesión $\SessionType$ \emph{finito}, la Definición \ref{def:pr}
describe un algoritmo recursivo para el cómputo de $\psfun\SessionType$. Cuando
$\SessionType$ es infinito (un protocolo recursivo) deja de ser tan obvio. Para
obtener un algoritmo que funcione en el caso general, interpretamos
$\psfun\SessionType$ como una \emph{variable aleatoria}. La Definición
\ref{def:pr} nos permite derivar un \emph{sistema finito de ecuaciones}
relacionando tales variables. El lado derecho de cada ecuación para
$\psfun\SessionType$ está expresada en términos de variables aleatorias que
corresponden a los nodos hijos en el árbol descrito por $\SessionType$. Como
$\SessionType$ tiene una cantidad finita de sub-árboles, obtenemos un número
finito de ecuaciones.

Luego, observamos que cada tipo sesión $\SessionType$ corresponde a una Cadena
de Markov en Tiempo Discreto (Discrete-Time Markov Chain, DTMC)~\todo{cita a
paper DTMC} cuyo espacio de estados es $\trees\SessionType = \mathset{S_1,
\dots, S_n}$ y tal que la probabilidad $p_{ij}$ de transicionar de un estado
$S_i$ al estado $S_j$ está dada por
\[
  p_{ij} \eqdef
  \begin{cases}
    p & \text{si $S_i \tred[p] S_j$}
    \\
    0 & \text{caso contrario}
  \end{cases}
\]
\label{pg:regreach}

\todo{Definición de regularidad y alcanzabilidad} La regularidad y alcanzabilidad
implican que la DTMC obtenida por el tipo sesión $\SessionType$ tiene un número
de estados finito y es absorbente. Esto significa siempre es posible alcanzar
un \emph{estado absorbente} ($\Done$ ó $\Idle$) desde cualquier \emph{estado
transiente} (cualquier otro tipo sesión). En cualquier DTMC absorbente con un
número de estados finito, la probabilidad de alcanzar un estado absorbente
desde uno transiente puede ser copmutado mediante la resolución de un sistema
de ecuaciones para el cual se garantiza una solución única
\todo{~cite{KemenySnell76}}.
