El desarrollo de software distribuido, ya sea en sus versiones tradicionales
como en sus variantes más recientes, tales como el \emph{Cloud/Fog Computing} o
la Internet de las cosas (IoT por su sigla en inglés), se enfocan
principalmente en la integración, cooperación y comunicación de componentes,
generalmente existentes y desarrollados independientemente.

Mas allá del desafío tecnológico que tal integración presupone, el
problema principal, desde el punto de vista del desarrollo, es razonar sobre
las propiedades de tales sistemas en términos de las propiedades de sus
componentes y de los mecanismos utilizados para su composición.

Esto último dio lugar al desarrollo de {\em técnicas de
descripción} de interfaces y de {\em soporte a nivel de lenguajes de
programación para el desarrollo de aplicaciones correctas por construcción}.
Los tipos comportamentales, tales como los tipos de sesión o {\em type-states}
constituyen un ejemplo paradigmático de esta línea de trabajo. La idea central
consiste en la utilización de sistemas de tipo (como extensión de los
utilizados en los lenguajes existentes) para garantizar estáticamente
propiedades relacionadas con la interacción de componentes.

Los últimos años vieron un auge en el desarrollo de tipos comportamentales
dentro de los cuales los {\em tipos de sesión} se mostraron como un formalismo
central para el estudio de aplicaciones distribuidas basadas en procesos que se
comunican a través de canales (introducción en~\cite{HuttelEtAl16}). A
continuación listamos artículos publicados en los úlitmos años en las mayores
conferencias del área~\cite{DBLP:journals/pacmpl/GhilezanPPSY21,
DBLP:journals/pacmpl/HinrichsenBK20, DBLP:journals/pacmpl/Castro-PerezY20,
DBLP:conf/ecoop/ImaiNYY19, DBLP:conf/ecoop/000119, DBLP:conf/esop/JongmansY20,
DBLP:conf/esop/VasconcelosCAM20, DBLP:conf/concur/Horne20,
DBLP:conf/concur/DasP20, DBHPS21,DBLP:conf/cc/Miu0Y021,
DBLP:journals/pacmpl/GriesemerHKLTTW20, DBLP:journals/pacmpl/MajumdarYZ20,
DBLP:conf/concur/InversoMPTT20}.

Los tipos de sesión fueron originalmente propuestos en~\cite {Honda93}, y
proponen estructurar la comunicación entre componentes alrededor
del concepto de {\em sesión}. Una {sesión} es un canal privado que permite
conectar a dos (a veces más) procesos, donde cada uno posee un
\emph{endpoint} de la sesión y lo debe usar siguiendo un protocolo bien
especificado, llamado \emph{tipo de sesión}, que restringe el secuencia de
mensajes que se pueden enviar y recibir a través de ese endpoint. Dependiendo
de si el mecanismo de tipado es estático o dinámico, el tipo sesión se utiliza
durante la compilación o la ejecución para asegurar que un programa utiliza el
canal de sesión de acuerdo a su tipo. Como en cualquier sistema de tipos, un
programa bien tipado goza de las propiedades que garantiza el sistema
propuesto. En general, se asegura la propiedad de  {\em type-safety} (es decir
los procesos intercambian información que tiene el tipo esperado), la
adherencia al protocolo (las secuencias de acciones sobre un canal siguen el
orden descripto por el protocolo), pero también pueden garantizar propiedades
como la ausencia de {\em deadlocks} y {\em livelocks}~\cite{HuttelEtAl16}.

En~\cite{DBLP:conf/concur/InversoMPTT20} se propuso el uso de tipos de sesión
para razonar probabilísticamente sobre propiedades de alcanzabilidad, donde
se dio una interpretación \emph{probabilística} a los operadores de
selección. Más específicamente, se pasa de una interpretación no
determinística a una probabilística y se estudia un sistema de tipos que
permita determinar la probabilidad con la que una sesión en particular
termina \emph{exitosamente}. Dado que no existe una interpretación universal
de ``terminación exitosa'', se diferencia la terminación exitosa de la
infructuosa a través de un {constructor dedicado}.

Si bien la línea trabajo sobre tipos sesión tiene raíces en resultados
teóricos sobre modelos fundacionales de la computación concurrente, como
los son CCS y el cálculo $\pi$, actualmente existen extensiones
concretas de sistemas de tipos de lenguajes de programación de escala
industrial, tales como Scala~\cite{DBLP:conf/pldi/ScalasYB19},
Dotty~\cite{DBLP:conf/pldi/ScalasYB19},
Go~\cite{DBLP:conf/icse/LangeNTY18,DBLP:conf/icse/LangeNTY18},
Rust~\cite{DBLP:journals/corr/abs-1909-05970,DBLP:conf/coordination/LagaillardieNY20},
Haskell~\cite{orchard2017session,DBLP:conf/haskell/LindleyM16},
OCaml~\cite{DBLP:journals/jfp/Padovani17,DBLP:conf/coordination/LagaillardieNY20,DBLP:conf/ecoop/ImaiNYY19},
Erlang~\cite{fowler2016erlang} y han sido aplicados, entre otros, al estudio de
{\em smart contracts} y tecnologías {\em
blockchain}~\cite{10.1145/3417516,DBLP:journals/corr/abs-1902-06056}. 

Este no es el caso para los tipos sesión probabilísticos, dado que la
propuesta en~\cite{DBLP:conf/concur/InversoMPTT20} presenta un sistema de
chequeo de tipos, pero deja abierta la definición de un algoritmo de inferencia
y su implementación en un lenguaje de programación. En~\cite{DasDH20} se
presentó una variante de tipos sesión probabilísticos
\emph{resource-aware}~\cite{DasHP18}, el mismo cuenta con la implementación de
un nuevo lenguaje desarrollado para demostrar las aplicaciones del sistema de
tipos pero no cuenta con una extensión para lenguajes de programación
existentes.

En este trabajo presentamos una implementación de el sistema de tipos
probabilísticos presentado en~\cite{DBLP:conf/concur/InversoMPTT20}. La misma
es una extensión de \FuSe~\cite{DBLP:journals/jfp/Padovani17}, una biblioteca
escrita en \OCaml que permite modelar tipos sesión mediante una codificación
que no depende de funcionalidades avanzadas del lenguaje anfitrión. La elección
fue motivada por la modularidad del proyecto y codificación que permitieron
una extensión al cálculo probabilístico sin perder las garantías del sistema
original.

\subparagraph*{Contribuciones y estructura del trabajo.}
\begin{itemize}

	\item Extendimos la codificación de tipos sesión presente en \FuSe para
		capturar la gramática de los tipos sesión probabilísticos y su
		combinación (Capítulo \ref{cap:tipos_sesion_prob}). La misma
		permite usufructuar el motor de inferencia estándar de \OCaml
		sin necesidad de modificarlo.

	\item Modificamos las primitivas de comunicación de \FuSe para
		incorporar elecciones probabilísticas y poder diferenciar una
		terminación exitosa de la fallida (Capítulo
		\ref{cap:implementacion}, Capítulo
		\ref{cap:elecciones_multi_sesion} y Capítulo
		\ref{cap:comp_sesiones_duales}). Se considerarán principalmente
		elecciones binarias (evitando el uso de elecciones
		generalizadas en término de variantes polimórficas).

	\item Utilizamos los tipos inferidos por el compilador anotados con
		elecciones probabilísticas para calcular la probabilidad de
		terminación exitosa de una sesión mediante el cómputo de la
		probabilidad de absorción de la cadena de Markov asociada al
		tipo de sesión inferido (Capítulo \ref{cap:prob_exito}).

	\item Implementamos un conjunto de ejemplos básicos, algunos presentes
		en~\cite{DBLP:conf/concur/InversoMPTT20}, para ilustrar la
		aplicabilidad de la propuesta.
\end{itemize}
