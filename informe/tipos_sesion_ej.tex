\section{Ejemplos básicos}

A continuación damos algunos ejemplos sencillos de esta interfaz utilizados
en~\cite{DBLP:journals/jfp/Padovani17}. El siguiente código implementa el
cliente de un servicio ``eco'' que espera un mensaje y lo devuelve al cliente.

\SimpleEchoClient

El argumento \OI{ep} tiene tipo $\Out\tvarA\In\tvarB\End$ y \OI{x} tipo
$\tvarA$. La función \OI{echo_client} comienza enviando un mensaje \OI{x} sobre
el endpoint \OI{ep}. La primitiva \OI{let} reasocia \OI{ep} al endpoint
retornado por \OI{send}, que ahora tiene tipo $\In\tvarB\End$.

El endpoint es después utilizado para recibir un mensaje de tipo \OI{'b} del
servicio. Finalmente, \OI{echo_client} cierra la sesión y devuelve el mensaje.

El servicio de eco está implementado por el \OI{echo_service} definido a
continuación, el mismo utiliza el argumento \OI{ep} de tipo
$\In\tvarA\Out\tvarA\End$ para recibir un mensaje \OI{x} y retornarlo al
cliente previo al cierre de sesión.

\SimpleEchoService

Podemos notar que $\Out\tvarA\In\tvarB\End$ no es el dual de
$\In\tvar\Out\tvar\End$ según la definición de dualidad presentada
anteriormente: para conectar el cliente y servidor, $\tvarB$ debe ser
\emph{unificado} con $\tvarA$. El código que conecta ambas partes es el
siguiente:

\SimpleEchoMain

Este código crea una nueva sesión cuyos endpoints se encuentran ligados a los
nombres \OI{a} y \OI{b}. Luego, lanza un thread que aplica \OI{echo_service} al
endpoint \OI{a}. Finalmente, aplica \OI{echo_client} al endpoint restante
\OI{b}.

Ahora deseamos generalizar el servicio para que un cliente pueda escoger
consumirlo o finalizar la interacción sin uso. Un servicio que ofrece esta
decisión se ilustra a continuación:

\OptionalEchoService

En este caso, el servicio utiliza la primitiva \OI{branch} para esperar la
etiqueta seleccionada por el cliente.

El tipo inicial de \OI{ep} es ahora $\Branch\set{\Tag[End] : \End, \Tag[Msg]
:\In\tvarA\Out\tvar\End}$ y el valor retornado por \OI{branch} es
$\set{\Tag[End] : \End, \Tag[Msg] :\In\tvarA\Out\tvar\End}$. En la rama
$\Tag[Msg]$ el servicio se comporta como antes. En $\Tag[End]$ el servicio
finaliza la sesión.

La siguiente función representa un posible cliente para \OI{opt_echo_service}:

\OptionalEchoClient

Esta función tiene tipo $\Choice\set{\Tag[End] : \End, \Tag[Msg] :
\Out\tvar\In\tvar\End} \to \tbool \to \tvar \to \tvar$ y su comportamiento depende del argumento booleano \OI{opt}.
