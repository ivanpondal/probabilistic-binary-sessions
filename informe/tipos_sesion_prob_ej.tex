\section{Ejemplos básicos}

Procedemos con la presentación de algunos ejemplos donde se aplican estas
nuevas primitivas. Para ello readaptaremos el servicio de eco presentado
anteriormente al modelo probabilístico.

\SimpleProbEchoService

Comenzamos con un servicio de eco que presenta dos caminos posibles: brindar
la función de eco o finalizar la comunicación. El argumento \OI{ep} tiene tipo
$\BinaryPBranch[p]{\In\tvarA\Out\tvarA\Done}{\Idle}$ donde la probabilidad $p$
queda libre hasta que el servicio sea unido a un cliente.

Para ello, podemos implementar el siguiente cliente que selecciona
determinísticamente la rama $\Tag[True]$.

\SimpleProbEchoClient

En este caso, el argumento \OI{ep} tiene tipo
$\BinaryPChoice[1]{\Out\tvarA\In\tvarB\Done}{\etvar}$ y \OI{x} tipo $\tvarA$.
Podemos observar que para el tipo sesión de este endpoint la probabilidad se
encuentra instanciada en $1$ acorde a la selección determinística mientras que
la rama $\Tag[False]$ queda libre haste ser asociada a su par.

Otra posibilidad es un cliente que selecciona determinísticamente la rama
$\Tag[False]$.

\SimpleIdleClient

Aquí el argumento \OI{ep} tiene tipo $\BinaryPChoice[0]{\etvar}{\Idle}$,
dejando libre la rama $\Tag[True]$.

Por úlitmo presentamos un cliente que mediante la primitiva \OI{pick}
selecciona con probabilidad $\frac{1}{2}$ una rama o la otra mediante
la composición de los clientes definidos en esta sección.

\SimpleCoinFlipEchoClient

El argumento \OI{ep} tiene tipo
$\BinaryPChoice[\frac{1}{2}]{\Out\tvarA\In\tvarB\Done}{\Idle}$. El código que
conecta este último cliente con el servicio de eco es el siguiente:

\SimpleCoinFlipEchoMain

Esta composición fuerza la unificación de tipos para los endpoints \OI{a} y
\OI{b} de forma tal que el endpoint \OI{a} asociado al servicio de eco toma el
tipo $\BinaryPBranch[\frac{1}{2}]{\In\tint\Out\tint\Done}{\Idle}$ y
su dual \OI{b}, asociado al cliente, recibe el tipo
$\BinaryPChoice[\frac{1}{2}]{\Out\tint\In\tint\Done}{\Idle}$.
