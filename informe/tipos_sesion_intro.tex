\section{Introducción}

Los tipos sesión se han consolidado como un formalismo para el
análisis modular de comunicación entre procesos. Una \emph{sesión} es un canal
privado que conecta dos procesos, cada uno con su conector o \emph{endpoint}
que habilita la comunicación estructurada mediante una especificación; el
\emph{tipo sesión}. A modo de ejemplo, el tipo sesión

\begin{equation}
    \label{eq:bare.auction}
    \T = \Out\tint{
        \BinaryPBranch[]\End{
            \In\tint{
                \BinaryPChoice[]\End\T
            }
        }
    }
\end{equation}

\noindent podría describir (parte de) un protocolo que debe seguir un proceso
comprador para participar en una subasta: el proceso debe enviar un valor entero
que representa su oferta ($ \Out [] \tint $) y esperar una decisión del
subastador. El protocolo puede proceder en este punto en dos formas distintas
correspondientes a las dos ramas del operador $\Branch$. El subastador puede
declarar que el artículo se vende (elección que representamos con $\Tag[True]$),
en cuyo caso la sesión termina inmediatamente ($ \End $), o no aceptar la oferta
(rama $\Tag[False]$) y enviar un contraoferta, recibida como un entero por el
comprador ($ \In [] \tint $). En ese momento, el comprador puede elegir
($\Choice$) terminar la subasta o reiniciar el protocolo con otra oferta, aquí
denotado por $\T$. Las elecciones posibles son representadas mediante etiquetas
(en este caso $\Tag[True]$ y $\Tag[False])$, $\Branch$ describe las elecciones
que puede recibir y $\Choice$ las que puede enviar.

\section{Gramática}
\label{sec:tipos_sesion_gram}

Formalmente la sintaxis de los tipos sesión está dada por la siguiente
gramática como fue presentada en~\cite{Melgratti2017AnOI}

\[
\begin{array}{@{}r@{~~}c@{~~}l@{}}
\TypeT, \TypeS & ::= &
\tbool
\rulemid \tint
\rulemid \tvar
\rulemid \SessionType
\rulemid \set[i\in I]{ \Tag_i : \Type_i }
\rulemid \cdots
\\
\SessionTypeT, \SessionTypeS & ::= &
\End
\rulemid \Out\Type\SessionType
\rulemid \In\Type\SessionType
\rulemid \Branch \set[i\in I]{\Tag_i : \SessionType_i}
\rulemid \Choice \set[i\in I]{\Tag_i : \SessionType_i}
\rulemid \etvar
\rulemid \dual\etvar
\end{array}
\]

donde $\TypeT$ y $\TypeS$ son utilizados para representar los tipos básicos,
variables libres, tipos sesión, unión disjunta y otros. Los tipos sesión quedan
descritos por $\SessionTypeT$ y $\SessionTypeS$. Una comunicación finalizada
queda descrita por el constructor de terminación ($\End$). Las operaciones de
escritura y lectura ($\Out\Type\SessionType$ y $\In\Type\SessionType$) describen
endpoints para enviar y recibir mensajes de tipo $\Type$ y continuar de acuerdo
con $\SessionType$. Las ramas ($\Branch \set[i\in I]{\Tag_i : \SessionType_i}$)
describen endpoints con varias continuaciones posibles ($\SessionType_i$)
identificadas por etiquetas ($\Tag_i$), el progreso queda dado por la elección
externa de una de ellas. Las elecciones ($\Choice \set[i\in I]{\Tag_i :
\SessionType_i}$) describen endpoints que pueden seleccionar mediante una
elección interna una etiqueta $\Tag_i$ y continuar como $\SessionType_i$. Como
consecuencia, la elección externa de una rama ($\Branch$) queda determinada por
su correspondiente elección interna ($\Choice$).

Por último, $\etvar$ y $\dual\etvar$ representan variables libres y su dual. El
\emph{dual} de un tipo sesión $\SessionType$, escrito como $\dual\SessionType$,
se obtiene intercambiando las operaciones de lectura y escritura. Esta operación
nos permite obtener para cualquier endpoint, el tipo sesión de su contraparte en
la comunicación que describe. La misma queda definida por las siguientes
ecuaciones:

\[
\begin{array}{c}
  \begin{array}{@{}r@{~~}c@{~~}l@{}}
    \smash{\dual{\dual\etvar}} & = & \etvar \\
    \dual\End & = & \End \\
  \end{array}
  \qquad
  \begin{array}{@{}r@{~~}c@{~~}l@{}}
    \dual{(\In\Type\SessionType)} & = & \Out\Type\dual\SessionType \\
    \dual{(\Out\Type\SessionType)} & = & \In\Type\dual\SessionType \\
  \end{array}
  \qquad
  \begin{array}{@{}r@{~~}c@{~~}l@{}}
    \dual{\Branch \set[i\in I]{\Tag_i : \SessionType_i}}
    & = & \Choice \set[i\in I]{\Tag_i : \dual{\SessionType_i}} \\
    \dual{\Choice \set[i\in I]{\Tag_i : \SessionType_i}}
    & = & \Branch \set[i\in I]{\Tag_i : \dual{\SessionType_i}} \\
  \end{array}
\end{array}
\]

Esta gramática permite describir también tipos sesión infinitos. Para ello
pedimos se cumplan las siguientes condiciones:
\begin{description}

	\item[Regularidad] Requerimos que todo árbol consista de un número
		finito de subárboles \emph{distinguibles}. Esta condición
		asegura que los tipos sesión sean representables mediante la
		notación $\mu$~\cite{Pierce02} o como soluciones a un
		conjunto finito de ecuaciones~\cite{Courcelle83}.

	\item[Alcanzabilidad] Requerimos que todo sub-árbol $\SessionType$ de
		un tipo sesión contenga una \emph{hoja alcanzable} etiquetada por
		$\End$. Esta condición asegura que siempre sea posible terminar una
		sesión independientemente de hace cuánto esté corriendo.
\end{description}
