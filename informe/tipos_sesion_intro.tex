\section{Introducción}

Los tipos sesión [TODO: citas] se han consolidado como un formalismo para el
análisis modular de comunicación entre procesos. Una \emph{sesión} es un canal
privado que conecta dos procesos, cada uno con su conector u \emph{endpoint} que
establece qué estructura tiene la comunicación mediante una especificación; el
\emph{tipo sesión}. A modo de ejemplo, el tipo sesión

\begin{equation}
    \label{eq:bare.auction}
    \cout \tint \parens {
        \tbranch [] \tend {
            \cin \tint \parens {
                \tchoice [] \tend \T
            }
        }
    }
\end{equation}

podría describir (parte de) un protocolo que debe seguir un proceso comprador
para participar en una subasta: el proceso debe enviar un valor entero que es su
oferta ($ \cout [] \tint $) y espera una decisión del subastador. El protocolo
puede proceder en este punto en dos formas distintas correspondientes a las dos
ramas del operador $ \tbranch [] {} {}$. El subastador puede declarar que el
artículo se vende, en cuyo caso la sesión termina inmediatamente ($ \tend $), o
puede contraofertar un nuevo valor ($ \cin [] \tint $). En ese momento, el
comprador puede elegir ($ \tchoice [] {} {} $) terminar la subasta o reiniciar
el protocolo con otra oferta, aquí denotado por $\T$.
