\section{Introducción}

Los tipos sesión se han consolidado como un formalismo para el
análisis modular de comunicación entre procesos. Una \emph{sesión} es un canal
privado que conecta dos procesos, cada uno con su conector o \emph{endpoint}
que habilita la comunicación estructurada mediante una especificación; el
\emph{tipo sesión}. A modo de ejemplo, el tipo sesión

\begin{equation}
    \label{eq:bare.auction}
    \T = \Out\tint{
        \BinaryPBranch[]\End{
            \In\tint{
                \BinaryPChoice[]\End\T
            }
        }
    }
\end{equation}

\noindent podría describir (parte de) un protocolo que debe seguir un proceso
comprador para participar en una subasta: el proceso debe enviar un valor entero
que representa su oferta ($ \Out [] \tint $) y esperar una decisión del
subastador. El protocolo puede proceder en este punto en dos formas distintas
correspondientes a las dos ramas del operador $\Branch$. El subastador puede
declarar que el artículo se vende (elección que representamos con $\Tag[True]$),
en cuyo caso la sesión termina inmediatamente ($ \End $), o no aceptar la oferta
(rama $\Tag[False]$) y enviar un contraoferta, recibida como un entero por el
comprador ($ \In [] \tint $). En ese momento, el comprador puede elegir
($\Choice$) terminar la subasta o reiniciar el protocolo con otra oferta, aquí
denotado por $\T$. Las elecciones posibles son representadas mediante etiquetas
(en este caso $\Tag[True]$ y $\Tag[False])$, $\Branch$ describe las elecciones
que puede recibir y $\Choice$ las que puede enviar.

\section{Gramática}

Formalmente la sintaxis de los tipos sesión está dada por la siguiente
gramática como fue presentada en~\cite{Melgratti2017AnOI}

\[
\begin{array}{@{}r@{~~}c@{~~}l@{}}
\TypeT, \TypeS & ::= &
\tbool
\rulemid \tint
\rulemid \tvar
\rulemid \SessionType
\rulemid \set[i\in I]{ \Tag_i : \Type_i }
\rulemid \cdots
\\
\SessionTypeT, \SessionTypeS & ::= &
\End
\rulemid \Out\Type\SessionType
\rulemid \In\Type\SessionType
\rulemid \Branch \set[i\in I]{\Tag_i : \SessionType_i}
\rulemid \Choice \set[i\in I]{\Tag_i : \SessionType_i}
\rulemid \etvar
\rulemid \dual\etvar
\end{array}
\]

donde $\TypeT$ y $\TypeS$ son utilizados para representar los tipos básicos,
variables libres, tipos sesión, unión disjunta y otros.
Los tipos sesión quedan descritos por $\SessionTypeT$ y $\SessionTypeS$ con su
constructor para marcar el cierre de un endpoint, operaciones de
lectura/escritura, ramas y elecciones así como variables libres y su dual.

El \emph{dual} de un tipo sesión $\SessionType$, escrito como
$\dual\SessionType$, se obtiene intercambiando las operaciones de lectura y
escritura. Lo definen las siguientes ecuaciones:

\[
\begin{array}{c}
  \begin{array}{@{}r@{~~}c@{~~}l@{}}
    \smash{\dual{\dual\etvar}} & = & \etvar \\
    \dual\End & = & \End \\
  \end{array}
  \qquad
  \begin{array}{@{}r@{~~}c@{~~}l@{}}
    \dual{(\In\Type\SessionType)} & = & \Out\Type\dual\SessionType \\
    \dual{(\Out\Type\SessionType)} & = & \In\Type\dual\SessionType \\
  \end{array}
  \qquad
  \begin{array}{@{}r@{~~}c@{~~}l@{}}
    \dual{\Branch \set[i\in I]{\Tag_i : \SessionType_i}}
    & = & \Choice \set[i\in I]{\Tag_i : \dual{\SessionType_i}} \\
    \dual{\Choice \set[i\in I]{\Tag_i : \SessionType_i}}
    & = & \Branch \set[i\in I]{\Tag_i : \dual{\SessionType_i}} \\
  \end{array}
\end{array}
\]
