Hasta ahora estudiamos programas que trabajan con una única sesión. La
introducción de múltiples sesiones requiere de algunas consideraciones respecto
cómo influyen las probabilidades de unas sobre otras. En esta sección veremos
cómo tales interacciones afectan nuestras primitivas y su tipado.

\section{Limitaciones con \OI{pick} de una sesión}

La primitiva de elección probabilística \OI{pick} opera sobre una única sesión,
tomando como argumento la misma y el comportamiento sobre cada rama posible.
Podría suceder que luego de una elección se interactue con otra sesión, a
continuación estudiaremos en detalle qué ocurre en tal escenario.

Comenzamos con un programa donde ambas ramas de la elección sobre \OI{epX}
presentan una comunicación con otra sesión \OI{epY}:

\TwoSessionsPickBothBranches

Aquí la sesión \OI{epX} tiene tipo $\BinaryPChoice[\frac{1}{2}]{\Done}{\Idle}$
mientras que \OI{epY} lleva el tipo $\Out\tbool\Done$. Por más que el \emph{valor}
comunicado por \OI{epY} dependa de la elección en \OI{epX}, el programa tipa
correctamente. Esto sucede porque el tipo de la sesión es idéntico en ambas
ramas.

Veamos qué ocurre cuando difiere el valor del tipo a transmitir sobre \OI{epY}
según la elección:

\TwoSessionsInvalidPickBothBranches

En este último caso, el sistema de tipo nos indicará que \OI{epY} presenta dos
tipos incompatibles entre sí: $\Out\tbool\Done$ y $\Out\tint\Done$. Lo que está
sucediendo es que el comportamiento de \OI{epY} difiere según la elección
realizada por \OI{epX} pero tal dependencia no es capturada por el sistema de
tipos.

Para poder describir estas relaciones es necesario introducir algunos aspectos
del sistema de tipos que se enfocan en el resultado de combinar sesiones en
elecciones.

\section{Reglas de tipado para elecciones}

\begin{definition}[probabilistic type combination]
  \label{def:ccomb}
  We write $\csum{t}{s}$ for the \emph{combination of $t$ and $s$
    weighed by $p$}, which is defined by cases on the form of $t$
  and $s$ as follows:
  \[
    \csum{t}{s}
    \eqdef
    \begin{cases}
      t & \text{si $t = s$}
      \\
      \BinaryPChoice[pq+(1-p)r]{\SessionTypeT}{\SessionTypeS} & \text{si $t = \BinaryPChoice[q]{\SessionTypeT}{\SessionTypeS}$} \\ & \text{y $s = \BinaryPChoice[r]{\SessionTypeT}{\SessionTypeS}$}
      \\
      \ClosedSessionType[pq+(1-p)r] & \text{si $t = \ClosedSessionType[q]$ y $s = \ClosedSessionType[r]$}
      \\
      \text{indefinido} & \text{caso contrario}
    \end{cases}
  \]
\end{definition}

