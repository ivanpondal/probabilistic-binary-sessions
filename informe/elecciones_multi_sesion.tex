Hasta ahora estudiamos programas que trabajan con una única sesión. La
introducción de múltiples sesiones requiere de algunas consideraciones respecto
cómo influyen las probabilidades de unas sobre otras. En esta sección veremos
cómo tales interacciones afectan nuestras primitivas y su tipado.

\section{Limitaciones con \OI{pick} de una sesión}

La primitiva de elección probabilística \OI{pick} opera sobre una única sesión,
tomando como argumento la misma y el comportamiento sobre cada rama posible.
Podría suceder que alguna u ambas ramas interactuen con otra sesión, a
continuación estudiaremos qué ocurre en cada escenario.

Comenzamos con un programa donde la rama $\Tag[False]$ de la elección presenta
una comunicación con otra sesión:

\TwoSessionsPickSingleBranch

Aquí la sesión \OI{epX} tiene tipo $\BinaryPChoice[\frac{1}{2}]{\Done}{\Idle}$
mientras que \OI{epY} lleva el tipo $\Out\tbool\Done$. Dado que \OI{epY} no
realiza ningún tipo de elección tipa correctamente, la sesión es independiente
de \OI{epX}.

Continuamos con una leve modificación al programa anterior, ahora ambas ramas
presentan la comunicación con \OI{epY}:

\TwoSessionsPickBothBranches

Tanto \OI{epX} como \OI{epY} preservan los tipos del ejemplo anterior. La
diferencia se encuentra en que ahora el \emph{valor} comunicado por \OI{epY} se
ve afectado por la elección. Dado que los tipos no han cambiado, este programa
continua tipando correctamente.

Ahora sí, veamos qué ocurre cuando difiere el valor del tipo transmitido por
\OI{epY} según la rama:

\TwoSessionsInvalidPickBothBranches

En este último caso, el sistema de tipo nos indicará que \OI{epY} presenta dos
tipos incompatibles entre sí: $\Out\tbool\Done$ y $\Out\tint\Done$. Lo que está
sucediendo es que el comportamiento de \OI{epY} difiere según la elección
realizada por \OI{epX}.

\TwoSessionsPickTwo
