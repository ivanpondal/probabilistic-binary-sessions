Hasta ahora estudiamos programas que trabajan con una única sesión. La
introducción de múltiples sesiones requiere de algunas consideraciones respecto
cómo influyen las probabilidades de unas sobre otras. En esta sección veremos
cómo tales interacciones afectan nuestras primitivas y su tipado.

\section{Limitaciones con \OI{pick} de una sesión}

La primitiva de elección probabilística \OI{pick} opera sobre una única sesión,
tomando como argumento la misma y el comportamiento sobre cada rama posible.
Podría suceder que luego de una elección se interactue con otra sesión, a
continuación estudiaremos en detalle qué ocurre en tal escenario.

Comenzamos con un programa donde ambas ramas de la elección sobre \OI{epX}
presentan una comunicación con otra sesión \OI{epY}:

\TwoSessionsPickBothBranches

Aquí la sesión \OI{epX} tiene tipo $\BinaryPChoice[\frac{1}{2}]{\Done}{\Idle}$
mientras que \OI{epY} lleva el tipo $\Out\tbool\Done$. Por más que el \emph{valor}
comunicado por \OI{epY} dependa de la elección en \OI{epX}, el programa tipa
correctamente. Esto sucede porque el tipo de la sesión es idéntico en ambas
ramas.

Veamos qué ocurre cuando difiere el valor del tipo a transmitir sobre \OI{epY}
según la elección:

\TwoSessionsInvalidPickBothBranches

En este último caso, el sistema de tipo nos indicará que \OI{epY} presenta dos
tipos incompatibles entre sí: $\Out\tbool\Done$ y $\Out\tint\Done$. Lo que está
sucediendo es que el comportamiento de \OI{epY} difiere según la elección
realizada por \OI{epX} pero tal dependencia no es capturada por el sistema de
tipos.

Para poder describir estas relaciones es necesario introducir algunos aspectos
del sistema de tipos que se enfocan en el resultado de combinar sesiones en
elecciones.

\section{Combinador de tipo probabilístico}

Como vimos en el último ejemplo surge la necesidad de capturar el tipo de una
sesión que diverge acorde una elección. Para esto introducimos un
\emph{combinador de tipo probabilístico} que permite combinar tipos ponderando
las distintas formas en las cuales se consume una sesión bajo una probabilidad
determinada.

\todo{Cambiar idioma de definición}
\begin{definition}[combinador de tipo probabilístico]
  \label{def:ccomb}
  Escribimos $\csum{t}{s}$ para la combinación de $t$ y $s$ ponderada por $p$,
	que se encuentra definida por casos sobre la forma de $t$ y $s$:
  \[
    \csum{t}{s}
    \eqdef
    \begin{cases}
      t & \text{si $t = s$}
      \\
      \BinaryPChoice[pq+(1-p)r]{\SessionTypeT}{\SessionTypeS} & \text{si $t = \BinaryPChoice[q]{\SessionTypeT}{\SessionTypeS}$} \\ & \text{y $s = \BinaryPChoice[r]{\SessionTypeT}{\SessionTypeS}$}
      \\
      \ClosedSessionType[pq+(1-p)r] & \text{si $t = \ClosedSessionType[q]$ y $s = \ClosedSessionType[r]$}
      \\
      \text{indefinido} & \text{caso contrario}
    \end{cases}
  \]
\end{definition}

Intuitivamente, $\csum{t}{s}$ describe un recurso que es utilizado acorde a $t$
con probabilidad $p$ y como $s$ con probabilidad $1 - p$. La combinación de $t$
y $s$ se encuentra definida únicamente cuando $t$ y $s$ tienen formas
``compatibles'', el caso trivial siendo cuando poseen mismo tipo. Los casos
interesantes se dan cuando $t$ y $s$ describen una elección o una sesión
completa. Para el primero, la combinación resulta en una nueva elección que
lleva como probabilidad la suma convexa de sus partes. Al combinar sesiones
completas se toma la suma convexa sobre la probabilidad de éxito de las
sesiones involucradas.

\section{Reglas de tipado para elecciones}

A continuación presentamos las reglas de tipado para elecciones del cálculo de
tipos sesión probabilísticos \todo{Agregar bibliografía} y cómo se emplea el
combinador.

Utilizaremos contextos de tipado para el seguimiento de variables presentes en
procesos. Un \emph{contexto de tipado} es un mapa de variables a tipos escrito
como $x_1 : t_1, \dots, x_n : t_n$. Vamos a definir $\ContextA$ y $\ContextB$ para
representar estos contextos, escribiendo $\EmptyContext$ para un contexto vacío,
$\dom\Context$ para el dominio de $\Context$ y $\ContextA, \ContextB$ como la unión de
los contextos $\ContextA$ y $\ContextB$ cuando $\dom\ContextA \cap \dom\ContextB =
\EmptyContext$. Además, extendemos $\csum{}{}$ punto a punto sobre contextos.

\begin{table}[htb]
\begin{gather*}
\inferrule[\rulename{t-branch}]{
  \wtp{\ContextA, x : T}P : \sigma
  \\
  \wtp{\ContextB, x : S}Q : \sigma
}{
  \wtp{\csum[p]\ContextA\ContextB, x : \BinaryPBranch{T}{S}}{
	  \OI{match} \ (\OI{branch} \ x) \ \OI{with} \
	  \Tag[True] \rightarrow Q \ | \ \Tag[False] \rightarrow P : \sigma}
}
\\\\
\inferrule[\rulename{t-choice}]{
  \wtp \EmptyContext prob : p
  \\
  \wtp\ContextA P : \sigma
  \\
  \wtp\ContextB Q : \sigma
}{
  \wtp{\csum\ContextA\ContextB}{\OI{pick}\ prob\ P\ Q} : \sigma
}
\end{gather*}
\caption{\label{tab:rules} Reglas de tipado.}
\end{table}

En Tabla \ref{tab:rules} adaptamos ligeramente las reglas \rulename{t-choice} y
\rulename{t-branch} con el equivalente a la sintaxis de nuestra extensión; las
primitivas \OI{branch} y \OI{pick}. La regla \rulename{t-branch} tipa un
proceso que recibe una elección mediante la sesión $x$ y continua como
$\SessionTypeT$ con probabilidad $p$, caso contrario $\SessionTypeS$. El tipo
de $x$ debe ser de la forma $\BinaryPBranch{T}{S}$, el resto de los recursos
son lo que nos compete en esta sección, ya que el proceso deberá comportarse
según $\ContextA$ en caso de ir por $\Tag[True]$ o $\ContextB$ si es
$\Tag[False]$.  De esta forma el comportamiento del proceso queda descrito por
$\csum[p]\ContextA\ContextB$, bajo la definición del combinador esto implica
que futuras elecciones en $\ContextA$ y $\ContextB$ serán ajustadas como
efecto de la elección recibida en $x$. La regla \rulename{t-choice} permite
tipar elecciones probabilísticas combinando la evolución de ambos contextos
mediante $\csum[p]{}{}$.

Retomando nuestro ejemplo anterior, podemos observar que el tipo de
\OI{epY} en cada rama de la elección no era compatible bajo la definición del
combinador. Veamos qué sucede al utilizar figuras compatibles:

\TwoSessionsInvalidPickBothBranchesWithSelect

Ahora \OI{epY} tiene tipo $\BinaryPChoice[0]{\SessionTypeT}{\Out\tbool\Done}$
en la rama $\Tag[False]$ y $\BinaryPChoice[1]{\Out\tint\Done}{\SessionTypeS}$
cuando se escoge $\Tag[True]$. Estos tipos son compatibles para la combinación,
sin embargo el programa no tipa ya que el motor de inferencia no dispone de la
información necesaria para unirlos. En la Tabla \ref{tab:rules} el combinador
probabilístico se aplica punto a punto sobre todas los recursos presentes en
ambos contextos. Nuestra primitiva \OI{pick} está limitada a combinar los tipos
de \OI{epX}.

\section{Extensión para elecciones multi-sesión}

Habiendo presentado las limitaciones del \OI{pick} actual y cómo estas son
resueltas en el cálculo probabilístico contamos con el contexto suficiente para
introducir nuestra solución. Luego de las últimas modificaciones a nuestro
programa, nos falta brindar al motor de inferencia la información necesaria
para poder combinar elecciones de más de un tipo sesión.

Idealmente, nos gustaría que la primitiva \OI{pick} permitiera la combinación
de cualquier número de elecciones; un desafío que presenta tal objetivo es el
de preservar las restricciones de tipo necesarias para evitar combinaciones
inválidas y poder calcular la probabilidad de éxito de las sesiones
involucradas. Considerando además la potencial complejidad de esta hipotética
interfaz, decidimos en cambio agregar primitivas específicas al número de
sesiones a combinar.

Para ello extendemos nuestro conjunto de primitivas comenzando con la
introducción de la primitiva \OI{pick_2ch} y su tipo en Tabla
\ref{tab:prob_api_pick_2ch}. Esta primitiva permite combinar las elecciones de
exactamente dos sesiones. Dada una probabilidad $p$, escoge $\Tag[True]$ con
probabilidad $p$ y $\Tag[False]$ con $p - 1$ sobre el endpoint con tipo
$\BinaryPChoice[p]{\etvarA}{\etvarB}$. Además, cada rama presenta una segunda
elección cuya combinación se ve reflejada en el endpoint con tipo
$\BinaryPChoice[pq + (1- p)r]{\etvarC}{\etvarD}$. Al igual que con \OI{pick} el
comportamiento sobre la rama seleccionada queda definido por las funciones que de
argumento reciben los tipos sesión correspondientes a la elección. Por ejemplo,
al escoger $\Tag[True]$ se llama a la función que toma la elección con $p = 1$
($\BinaryPChoice[1]{\etvarA}{\etvarB}$) y la sesión secundaria con probabilidad
$q$ ($\BinaryPChoice[q]{\etvarC}{\etvarD}$).

\begin{table}[htb]
\begin{OCamlD}[frame=single]
	val pick_2ch:
	    $p$ -> $(\BinaryPChoice[0]{\etvarA}{\etvarB}$ -> $\BinaryPChoice[r]{\etvarC}{\etvarD}$ -> $\alpha)$
	      -> $(\BinaryPChoice[1]{\etvarA}{\etvarB}$ -> $\BinaryPChoice[q]{\etvarC}{\etvarD}$ -> $\alpha)$
	      -> $\BinaryPChoice[p]{\etvarA}{\etvarB}$ -> $\BinaryPChoice[pq + (1 - p)r]{\etvarC}{\etvarD}$
	      -> $\alpha$
\end{OCamlD}
\caption{Interfaz de primitiva \OI{pick_2ch}}
\label{tab:prob_api_pick_2ch}
\end{table}
\todo{Arreglar línea blanca en definición e pick 2}

Veamos cómo queda nuestro ejemplo original utilizando esta nueva primitiva:

\TwoSessionsPickTwo

Ya no tenemos más problemas con el tipado de \OI{epY}, en las funciones que
definen el comportamiento de cada rama la sesión lleva su tipo como argumento
($\BinaryPChoice[0]{\Out\tint\Done}{\Out\tbool\Done}$ y
$\BinaryPChoice[1]{\Out\tint\Done}{\Out\tbool\Done}$). A su vez, la combinación
de ambos tipos queda reflejada en el tipo
$\BinaryPChoice[\frac{1}{2}]{\Out\tint\Done}{\Out\tbool\Done}$ simulando así la
aplicación del combinador probabilístico.

Esta solución puede extenderse a más de dos sesiones siguiendo la misma idea de
materializar la combinación probabilística en el tipo de la primitiva.
