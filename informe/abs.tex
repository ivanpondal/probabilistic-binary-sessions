\chapter*{\runtitulo}

Los últimos años testimonian un auge en el desarrollo de técnicas de
descripción de interfaces y soporte a nivel de lenguajes de
programación para el desarrollo de aplicaciones correctas por construcción. El
desarrollo de tipos comportamentales y, en especial, los tipos de sesión
se han consolidado como un formalismo central para el análisis modular de
aplicaciones distribuidas basadas en procesos que comunican a través de canales.

En~\cite{DBLP:conf/concur/InversoMPTT20} se propuso el uso de tipos de sesión
para razonar probabilísticamente sobre propiedades de alcanzabilidad. Aquel
trabajo presenta un sistema de chequeo de tipos que permite determinar la
probabilidad con la que una sesión en particular termina \emph{exitosamente}.

La investigación presenta una implementación del sistema de tipos
probabilísticos presentado en~\cite{DBLP:conf/concur/InversoMPTT20} extendiendo
FuSe~\cite{DBLP:journals/jfp/Padovani17}, una implementación de tipos sesión
escrita en OCaml.

\bigskip

\noindent\textbf{Palabras claves:} TODO: etiquetas (no menos de 5).