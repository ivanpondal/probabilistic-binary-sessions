La implementación toma de base el trabajo realizado en \FuSe. En esta sección
presentaremos los cambios necesarios en la codificación y tipado de las
primitivas para modelar elecciones probabilísticas y cualquier información
adicional necesaria para el cálculo de terminación exitosa de un tipo sesión.

\section{Codificación}

\begin{center}
$\displaystyle
  \begin{array}{@{}c@{}}
    \textbf{Codificación de tipos sesión probabilísticos} \\
    \hline
    \hline
    \begin{array}[t]{@{}r@{~}c@{~}l@{}}
      \encfun\Done
      & = &
      \tsession\DoneBottomType\DoneBottomType
      \\
      \encfun\Idle
      & = &
      \tsession\tbottom\tbottom
      \\
      \encfun{\In\Type\SessionType}
      & = &
      \tsession{\encfun\Type\tmul\encfun\SessionType}\tbottom
      \\
      \encfun{\Out\Type\SessionType}
      & = &
      \tsession\tbottom{\encfun\Type\tmul\encfun{\dual\SessionType}}
      \\
      \encfun{\BinaryPBranch[p]{\SessionTypeT}{\SessionTypeS}}
      & = &
      \tsession{\BinaryLabels{\encfun{\SessionTypeT}}{\encfun{\SessionTypeS}}\tmul p}\tbottom
      \\
      \encfun{\BinaryPChoice[p]{\SessionTypeT}{\SessionTypeS}}
      & = &
      \tsession{\tbottom}{\BinaryLabels{\encfun{\dual\SessionTypeT}}{\encfun{\dual\SessionTypeS}}\tmul p}
      \\
      \encfun\etvar
      & = &
      \tsession{\renc\etvar}{\senc\etvar}
      \\
      \encfun{\sdual\etvar}
      & = &
      \tsession{\senc\etvar}{\renc\etvar}
    \end{array}
  \end{array}
$
\end{center}

\subsection{Codificación de probabilidad}

Podemos notar las elecciones llevan en su codificación una probabilidad
asociada $p$ que también está presente como argumento en la primitiva
\OI{pick}. Comenzamos definiendo $p$ como la probabilidad con la que se escoge
la rama $\Tag[True]$.
\begin{align*}
	P(\Tag[True]) &= p \\
	P(\Tag[False]) &= 1-p\\
	0 \le p \le 1 &,\ p \in \mathbb{R}
\end{align*}

Normalmente podría utilizarse una representación de coma flotante como \tfloat,
el problema con tal modelo es que no brinda información del valor de $p$ en
tiempo de compilación. Disponer del valor a nivel tipo permite la definición de
interfaces más fuertes como en \OI{pick}, donde la probabiliad $p$ que se
recibe como argumento debe coincidir con la utilizada en el tipo sesión sobre
el cual se opera.

\begin{table}[htb]
\begin{OCamlD}[frame=single]
	val pick: $\bm{p}$ -> $(\BinaryPChoice[0]{\etvarA}{\etvarB}$ -> $\alpha)$
	            -> $(\BinaryPChoice[1]{\etvarA}{\etvarB}$ -> $\alpha)$
		    -> $\BinaryPChoice[\bm{p}]{\etvarA}{\etvarB}$ -> $\alpha$
\end{OCamlD}
\caption{Interfaz de primitiva \OI{pick}}
\label{tab:prob_api_pick}
\end{table}

Otra ventaja es la de poder utilizar dicho valor en cualquier tipo de análisis
estático, requisito para el cálculo de la probabilidad con la que una sesión
termina exitosamente.

\subsection{Representando naturales y racionales}

En este trabajo decidimos representar $p$ como un racional cuya construcción
queda presente en el tipo. Para ello primero escribimos una definición de tipo
para los naturales que se basa en la construcción mediante la función sucesor.

\begin{table}[htb]
\begin{OCamlD}[frame=single]
      type _z
      type _s
      type _ nat = Z : zero nat | S : 'a nat -> (_s * 'a) nat
\end{OCamlD}
\caption{Representación para naturales y el cero}
\label{tab:def_nat}
\end{table}

Primero definimos los tipos \OI{_z} y \OI{_s} que serán utilizados como
etiquetas para distinguir a nivel tipo entre cero y la función sucesor.
Luego definimos \OI{nat} como una estructura de datos algebraica particular
denominada GADT (Generalized Algebraic Data Type) presente en \OCaml que
permite la implementación de restricciones en los parámetros de sus
constructores. Aquí definimos un constructor \OI{Z} que actua como cero y otro
\OI{S} que hace de función sucesor incrementando en uno el natural que toma
como argumento.

\begin{table}[htb]
\begin{OCamlD}[frame=single]
      let zero : _z nat = Z
      let one : (_s * _z) nat = S Z
      let two : (_s * (_s * _z)) nat = S (S Z)
\end{OCamlD}

\caption{Construcción de naturales}
\label{tab:ex_nat}
\end{table}

Teniendo una representación para naturales, definimos \OI{frac} como otro GADT
que permite la construcción de números racionales. En este caso se utiliza una
tupla donde la primer componente representa al numerador y la segunda el
denominador, prohibiendo el cero mediante una restricción de tipo.

\begin{table}[htb]
\begin{OCamlD}[frame=single]
	type _ frac = Fraction : 'a nat * (_s * 'b) nat ->
	                        ('a nat * (_s * 'b) nat) frac

	let one_half : ((_s * _z) nat *
	                (_s * (_s * _z)) nat) frac
	                = Fraction (S Z, S (S Z))

\end{OCamlD}
\caption{Representación y ejemplo de racionales}
\label{tab:def_rational}
\end{table}
