La implementación toma de base el trabajo realizado en \FuSe. En esta sección
presentaremos los cambios necesarios en la codificación y tipado de las
primitivas para modelar elecciones probabilísticas y cualquier información
adicional necesaria para el cálculo de terminación exitosa de un tipo sesión.

\section{Codificación}

\begin{center}
$\displaystyle
  \begin{array}{@{}c@{}}
    \textbf{Codificación de tipos sesión probabilísticos} \\
    \hline
    \hline
    \begin{array}[t]{@{}r@{~}c@{~}l@{}}
      \encfun\Done
      & = &
      \tsession\DoneBottomType\DoneBottomType
      \\
      \encfun\Idle
      & = &
      \tsession\tbottom\tbottom
      \\
      \encfun{\In\Type\SessionType}
      & = &
      \tsession{\encfun\Type\tmul\encfun\SessionType}\tbottom
      \\
      \encfun{\Out\Type\SessionType}
      & = &
      \tsession\tbottom{\encfun\Type\tmul\encfun{\dual\SessionType}}
      \\
      \encfun{\BinaryPBranch[p]{\SessionTypeT}{\SessionTypeS}}
      & = &
      \tsession{\BinaryLabels{\encfun{\SessionTypeT}}{\encfun{\SessionTypeS}}\tmul p}\tbottom
      \\
      \encfun{\BinaryPChoice[p]{\SessionTypeT}{\SessionTypeS}}
      & = &
      \tsession{\tbottom}{\BinaryLabels{\encfun{\dual\SessionTypeT}}{\encfun{\dual\SessionTypeS}}\tmul p}
      \\
      \encfun\etvar
      & = &
      \tsession{\renc\etvar}{\senc\etvar}
      \\
      \encfun{\sdual\etvar}
      & = &
      \tsession{\senc\etvar}{\renc\etvar}
    \end{array}
  \end{array}
$
\end{center}
