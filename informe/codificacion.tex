La implementación toma de base el trabajo realizado en \FuSe. En esta sección
presentaremos los cambios necesarios en la codificación y tipado de las
primitivas para modelar elecciones probabilísticas y cualquier información
adicional necesaria para el cálculo de terminación exitosa de un tipo sesión.

\section{Codificación}

\FuSe toma la codificación propuesta en \cite{Dardha}
posteriormente refinada~\cite{DBLP:journals/jfp/Padovani17}. La
idea principal es que una secuencia de mensajes sobre una sesión puede
modelarse como una serie de comunicaciones mediante canales de un uso. En este
modelo cada mensaje lleva un valor acompañado de un canal fresco sobre el cual
continuar la comunicación.

Esta codificación depende de los siguientes tipos:

\begin{itemize}
	\item $\tbottom$ representa la ausencia de valor e identifica la
		terminación no exitosa.
	\item $\DoneBottomType$ identifica la terminación exitosa.
	\item $\tsession{\renc{}}{\senc{}}$ describe canales que
		reciben mensajes del tipo $\renc{}$ y envían de tipo $\senc{}$.
		Tanto $\renc{}$ como $\senc{}$ pueden ser instanciados con
		$\tbottom$ para indicar que no se recibe y/o envía ningún
		mensaje.
\end{itemize}

La relación entre los tipos sesión $\SessionType$ y los tipos de la forma
$\tsession\TypeT\TypeS$ está dada por la función $\encfun\cdot$ definida debajo

\begin{center}
$\displaystyle
  \begin{array}{@{}c@{}}
    \textbf{Codificación de tipos sesión probabilísticos} \\
    \hline
    \hline
    \begin{array}[t]{@{}r@{~}c@{~}l@{}}
      \encfun\Done
      & = &
      \tsession\DoneBottomType\DoneBottomType
      \\
      \encfun\Idle
      & = &
      \tsession\tbottom\tbottom
      \\
      \encfun{\In\Type\SessionType}
      & = &
      \tsession{\encfun\Type\tmul\encfun\SessionType}\tbottom
      \\
      \encfun{\Out\Type\SessionType}
      & = &
      \tsession\tbottom{\encfun\Type\tmul\encfun{\dual\SessionType}}
      \\
      \encfun{\BinaryPBranch[p]{\SessionTypeT}{\SessionTypeS}}
      & = &
      \tsession{\BinaryLabels{\encfun{\SessionTypeT}}{\encfun{\SessionTypeS}}\tmul p}\tbottom
      \\
      \encfun{\BinaryPChoice[p]{\SessionTypeT}{\SessionTypeS}}
      & = &
      \tsession{\tbottom}{\BinaryLabels{\encfun{\dual\SessionTypeT}}{\encfun{\dual\SessionTypeS}}\tmul p}
      \\
      \encfun\etvar
      & = &
      \tsession{\renc\etvar}{\senc\etvar}
      \\
      \encfun{\sdual\etvar}
      & = &
      \tsession{\senc\etvar}{\renc\etvar}
    \end{array}
  \end{array}
$
\end{center}

extendida homomórficamente a todos los tipos. Asumimos que para cada variable
de tipo sesión $\etvar$ existen dos variables de tipo $\senc\etvar$ y
$\renc\etvar$ distintas entre sí y de cualquier tipo de variable $\tvar$.

Como ejemplo, el tipo sessión $\In\tvar\etvar$ se codifica como
$\tsession{\tvar \tmul \tsession{\renc\etvar}{\senc\etvar}}\tbottom$,
describiendo un canal para recibir un mensaje de tipo $\tvar \tmul
\tsession{\renc\etvar}{\senc\etvar}$; una componente de tipo $\tvar$ (el valor a
comunicar) y otra de tipo $\tsession{\renc\etvar}{\senc\etvar}$ (el canal de
continuación sobre el cual avanza la comunicación).

Para la codificación de los tipos sesión que envían un valor o una elección
tenemos la particularidad de que la continuación está dualizada. Esto se debe a
que el tipo asociado a la continuación del canal describe el comportamiento del
\emph{receptor}. Además, este detalle permite una forma sencilla de expresar la
dualidad entre tipos sesión aun cuando no se encuentran completamente
instanciados.

Como ejemplo, tomemos la codificación de
$\SessionType = \BinaryPChoice[p]{\Tag[\Out\tvarA\In\tvarB\Done]}{\Tag[\Idle]}$
\[
\begin{array}{@{}rcl@{}}
  \encfun\SessionType
  & = & \tsession\tbottom{
	  \BinaryLabels{\encfun{\In\tvarA\Out\tvarB\Done}}{\encfun\Idle}\tmul p
    }
  \\
  & = & \tsession\tbottom{
	  \BinaryLabels{\tsession{\tvarA\tmul\encfun{\Out\tvarB\Done}}{\tbottom}}
	               {\tsession\tbottom\tbottom}\tmul p
  }
  \\
  & = & \tsession\tbottom{
	  \BinaryLabels{\tsession{\tvarA\tmul\tsession{\tbottom}
	                                              {\tvarB\tmul\encfun{\Done}}}
	                         {\tbottom}}
	               {\tsession\tbottom\tbottom}\tmul p
  }\\
  & = & \tsession\tbottom{
	  \BinaryLabels{\tsession{\tvarA\tmul\tsession{\tbottom}
						      {\tvarB\tmul\tsession{\DoneBottomType}
	                                                                   {\DoneBottomType}}}
	                         {\tbottom}}
	               {\tsession\tbottom\tbottom}\tmul p
  }
\end{array}
\]

Luego la de su dual
$\dual\SessionType = \BinaryPBranch{\In\tvarA\Out\tvarB\End}{\Idle}$
\[
\begin{array}{@{}rcl@{}}
  \encfun{\dual\SessionType}
  & = & \tsession{
	  \BinaryLabels{\encfun{\In\tvarA\Out\tvarB\Done}}{\encfun\Idle}\tmul p
    }\tbottom
  \\
  & = & \tsession{
	  \BinaryLabels{\tsession{\tvarA\tmul\encfun{\Out\tvarB\Done}}{\tbottom}}
	               {\tsession\tbottom\tbottom}\tmul p
  }\tbottom
  \\
  & = & \tsession{
	  \BinaryLabels{\tsession{\tvarA\tmul\tsession{\tbottom}
	                                              {\tvarB\tmul\encfun{\Done}}}
	                         {\tbottom}}
	               {\tsession\tbottom\tbottom}\tmul p
  }\tbottom
  \\
  & = & \tsession{
	  \BinaryLabels{\tsession{\tvarA\tmul\tsession{\tbottom}
						      {\tvarB\tmul\tsession{\DoneBottomType}
	                                                                   {\DoneBottomType}}}
	                         {\tbottom}}
	               {\tsession\tbottom\tbottom}\tmul p
  }\tbottom
\end{array}
\]

Con este último ejemplo podemos notar que la codificación de
$\dual\SessionType$ puede obtenerse de $\SessionType$ permutando las
componentes de su par codificado. Esta es una propiedad que se cumple siempre
para esta codificación:

\begin{theorem}
  \label{thm:duality}
  Si $\encfun\SessionType = \tsession\TypeT\TypeS$, entonces
  $\encfun{\dual\SessionType} = \tsession\TypeS\TypeT$.
\end{theorem}
\begin{proof}
Sigue mutatis mutandis la prueba original en~\cite{Dardha}.
\end{proof}

De forma equivalente podemos decir que si
$\encfun\SessionTypeT = \tsession{\Type_1}{\Type_2}$ y
$\encfun\SessionTypeS = \tsession{\TypeS_1}{\TypeS_2}$, entonces
\[
\SessionTypeT = \dual\SessionTypeS
\iff
\encfun\SessionTypeT = \encfun{\dual\SessionTypeS}
\iff
\TypeT_1 = \TypeS_2
\wedge
\TypeT_2 = \TypeS_1
\]
Al igual que \FuSe, esto permite reducir el cáculo del dual de un tipo sesión a
la equivalencia entre tipos. También vale para tipos sesión desconocidos o
parcialmente instanciados: $\encfun\etvar =
\tsession{\renc\etvar}{\senc\etvar}$ y $\encfun{\sdual\etvar} =
\tsession{\senc\etvar}{\renc\etvar}$.

\begin{table}[htb]
	\begin{OCamlD}[frame=single]
  type _0
  type _1
  type $p_0$ (* alias para codificación de probabilidad cero *)
  type $p_1$ (* alias para codificación de probabilidad uno *)
  type ('r,'s) pst (* sintáxis en OCaml para $\tsession\tvarR\tvarS$ *)
  val create  : unit -> ('r,'s) pst * ('s,'r) pst
  val close   : (_0,_0) pst -> unit
  val idle    : (_1,_1) pst -> unit
  val send    : 'a -> (_0,('a * ('s,'r) pst)) pst -> ('r,'s) pst
  val receive : (('a * ('r,'s) pst),_0) pst -> 'a * ('r,'s) pst
  val select_true  : (_0,[`True of ('r,'s) pst |
                         `False of ('c,'d) pst] * $p_1$)
                      -> ('r,'s) pst
  val select_false : (_0,[`True of ('r,'s) pst |
                         `False of ('c,'d) pst] * $p_0$)
                      -> ('c,'d) pst
  val pick   : $p$ ->
             ((_0,[`True of ('r,'s) pst | `False of ('c,'d) pst]
                  * $p_0$) -> 'a) ->
             ((_0,[`True of ('r,'s) pst | `False of ('c,'d) pst]
                  * $p_1$) -> 'a) ->
             (_0,[`True of ('r,'s) pst | `False of ('c,'d) pst]
                 * $p$) -> 'a
  val branch : ([`True of ('r,'s) pst |
                 `False of ('c,'d) pst] * $p$,_0)
                -> [> `True of ('r,'s) pst | `False of ('c,'d) pst]
	\end{OCamlD}
	\caption{Interfaz \OCaml para tipos sesión probabilísticos.}
	\label{tab:signature}
\end{table}

Habiendo escogido la representación para tipos sesión probabilísticos podemos
observar la implementación de la interfaz \OCaml en la Tabla
\ref{tab:signature}. Existe una correspondencia directa entre la firma de las
funciones de la Tabla \ref{tab:signature} y las primitivas introducidas en
Tabla \ref{tab:prob_api}. Queda en evidencia que esta codificación dificulta
la lectura de los tipos sesión. Este problema se agudiza a medida que los
protocolos implementados son más complejos. Por este motivo extendimos
\texttt{rosetta}, una herramienta auxiliar que acompaña \FuSe, para imprimir
los tipos utilizando la notación presente en la gramática inicial. Al momento
de presentar tipos sesión inferidos por \OCaml utilizaremos el formato generado
por \texttt{rosetta} para facilitar la lectura.

\subsection{Codificación de probabilidad}

Podemos notar las elecciones llevan en su codificación una probabilidad
asociada $p$ que también está presente como argumento en la primitiva
\OI{pick}. Comenzamos definiendo $p$ como la probabilidad con la que se escoge
la rama $\Tag[True]$.
\begin{align*}
	P(\Tag[True]) &= p \\
	P(\Tag[False]) &= 1-p\\
	0 \le p \le 1 &,\ p \in \mathbb{R}
\end{align*}

Normalmente podría utilizarse una representación de coma flotante como \tfloat,
el problema con tal modelo es que no brinda información del valor de $p$ en
tiempo de compilación. Disponer del valor a nivel tipo permite la definición de
interfaces más fuertes como en \OI{pick}, donde la probabiliad $p$ que se
recibe como argumento debe coincidir con la utilizada en el tipo sesión sobre
el cual se opera.

\begin{table}[htb]
\begin{OCamlD}[frame=single]
	val pick: $\bm{p}$ -> $(\BinaryPChoice[0]{\etvarA}{\etvarB}$ -> $\alpha)$
	            -> $(\BinaryPChoice[1]{\etvarA}{\etvarB}$ -> $\alpha)$
		    -> $\BinaryPChoice[\bm{p}]{\etvarA}{\etvarB}$ -> $\alpha$
\end{OCamlD}
\caption{Interfaz de primitiva \OI{pick}}
\label{tab:prob_api_pick}
\end{table}

Otra ventaja es la de poder utilizar dicho valor en cualquier tipo de análisis
estático, requisito para el cálculo de la probabilidad con la que una sesión
termina exitosamente.

\subsection{Representando naturales y racionales}

En este trabajo decidimos representar $p$ como un racional cuya construcción
queda presente en el tipo. Para ello primero escribimos una definición de tipo
para los naturales que se basa en la construcción mediante la función sucesor.

\begin{table}[htb]
\begin{OCamlD}[frame=single]
      type _z
      type _s
      type _ nat = Z : zero nat | S : 'a nat -> (_s * 'a) nat
\end{OCamlD}
\caption{Representación para naturales y el cero}
\label{tab:def_nat}
\end{table}

Primero definimos los tipos \OI{_z} y \OI{_s} que serán utilizados como
etiquetas para distinguir a nivel tipo entre cero y la función sucesor.
Luego definimos \OI{nat} como una estructura de datos algebraica particular
denominada GADT (Generalized Algebraic Data Type) presente en
\OCaml~\cite{YallopM} que permite la implementación de restricciones en los
parámetros de sus constructores. Aquí definimos un constructor \OI{Z} que actua
como cero y otro \OI{S} que hace de función sucesor incrementando en uno el
natural que toma como argumento.

\begin{table}[htb]
\begin{OCamlD}[frame=single]
      let zero : _z nat = Z
      let one : (_s * _z) nat = S Z
      let two : (_s * (_s * _z)) nat = S (S Z)
\end{OCamlD}

\caption{Construcción de naturales}
\label{tab:ex_nat}
\end{table}

Teniendo una representación para naturales, definimos \OI{frac} como otro GADT
que permite la construcción de números racionales. En este caso se utiliza una
tupla donde la primer componente representa al numerador y la segunda el
denominador, prohibiendo el cero mediante una restricción de tipo.

\begin{table}[htb]
\begin{OCamlD}[frame=single]
	type _ frac = Fraction : 'a nat * (_s * 'b) nat ->
	                        ('a nat * (_s * 'b) nat) frac

	let one_half : ((_s * _z) nat *
	                (_s * (_s * _z)) nat) frac
	                = Fraction (S Z, S (S Z))

\end{OCamlD}
\caption{Representación y ejemplo de racionales}
\label{tab:def_rational}
\end{table}
