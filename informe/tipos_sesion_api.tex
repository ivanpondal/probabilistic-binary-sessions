\section{Interfaz programática}

\begin{table}[htb]
    \begin{OCamlD}[frame=single]
        val create  : unit -> $\etvar$ * $\sdual\etvar$
        val close   : $\End$ -> unit
        val send    : $\tvar$ -> $\Out\tvar\etvar$ -> $\etvar$
        val receive : $\In\tvar\etvar$ -> $\tvar$ * $\etvar$
        val select  : ($\sdual{\etvar_k}$ -> $\set[i\in I]{\Tag_i : \sdual{\etvar_i}}$) -> $\Choice \set[i\in I]{\Tag_i : \etvar_i}$ -> $\etvar_k$
        val branch  : $\Branch \set[i\in I]{\Tag_i : \etvar_i}$ -> $\set[i\in I]{\Tag_i : \etvar_i}$
    \end{OCamlD}
    \caption{Interfaz programática para tipos sesión}
    \label{tab:api}
\end{table}

En~\cite{DBLP:journals/jfp/Padovani17} se presentó el desarrollo de \FuSe, una
implementación en \OCaml de la interfaz de tipos sesión definida en
Tabla~\ref{tab:api}.

\begin{itemize}
	\item \OI{create}: crea una nueva sesión y devuelve un par de endpoints con
tipos sesión duales.
	\item \OI{close}: señaliza el fin de una sesión.
	\item \OI{send}: envía mensaje de tipo $\tvar$ sobre endpoint con tipo
		$\Out\tvar\etvar$, retorna endpoint con tipo $\etvar$ para
		denotar el avance en la comunicación.
	\item \OI{receive}: espera recibir mensaje de tipo $\tvar$ sobre
		endpoint con tipo $\In\tvar\etvar$, retorna un par con
		mensaje y endpoint de tipo $\etvar$.
\end{itemize}

Las primitivas \OI{branch} y \OI{select} permiten tratar con sesiones que
presentan caminos con interacciones alternativas, cada camino siendo
identificado por una \emph{etiqueta} $\TagVar_i$.
		
\begin{itemize}
	\item \OI{select}: envía etiqueta $\TagVar_k$ con $k\in I$
		sobre endpoint con tipo $\Choice \set[i\in I]{\Tag_i :
		\etvar_i}$, retorna un endpoint con el tipo $\etvar_k$ asociado
		a la etiqueta seleccionada. En \OCaml, el etiquetado es
		representado mediante una función que \emph{inyecta} un
		endpoint (ejemplo, el tipo $\sdual{\etvar_k}$) en una unión
		disjunta de tipo $\set[i\in I]{\Tag_i : \sdual{\etvar_i}}$ donde $k\in
		I$.
	\item \OI{branch}: dualmente, espera recibir la comunicación de una
		elección en un endpoint de tipo $\Branch \set[i\in I]{\Tag_i :
		\etvar_i}$, retorna la elección recibida como endpoint
		inyectado a la unión disjunta descrita por su tipo.
\end{itemize}
