Consideramos que con el ague en el desarrollo de técnicas de descripción de
interfaces y soporte a nivel de lenguajes de programación que intenten asegurar
la corrección por construcción, esta clase de trabajos enfocados en el
desarrollo de interfaces extensibles e implementaciones sencillas pueden
contribuir a la popularización y uso de tipos comportamentales como los tipos
sesión.

Inspirados por el trabajo de \FuSe y su codificación de tipos sesión, hemos
presentado una extensión que implementa el cálculo de tipos sesión
probabilísticos sin perder las garantías del sistema original. Nuestra
implementación permite la representación e inferencia de tipos sesión
probabilísticos con una interfaz programática enfocada en la usabilidad del
sistema. La misma no requiere de funcionalidades avanzadas del lenguaje de
implementación como lo pueden ser el uso de mónadas o sistemas de tipo
sub-estructurales. El uso de mónadas~\cite{fmt18}~\cite{pucella08} como
sesiones de linealidad es más fuerte que nuestra validación mediante tipos y
tiempo de ejecución heredada de \FuSe, sin embargo, consideramos la primera
sacrifica expresividad y/o usabilidad del sistema.

Existen varias áreas donde podría continuar esta rama de trabajo. Una de ellas
es la implementación de distribuciones de probabilidad para tratar elecciones
con más de una rama. La representación de naturales y fracciones como tipos no
restringe al uso de valores mayores a 1, esto puede validarse en tiempo de
ejecución, al momento de utilizar la herramienta de decodificación o con alguna
mejora a la representación actual que sea habilitada por el sistema de tipos.
Al momento de calcular la probabilidad de éxito de una sesión, si esta no tiene
instanciadas todas las probabilidades involucradas, el decodificador de tipos
podría utilizar alguna representación simbólica para variables libres y
retornar un resultado en función de las mismas.
