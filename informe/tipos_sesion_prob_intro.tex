En~\cite{DBLP:conf/concur/InversoMPTT20} se propuso el uso de tipos de sesión
para razonar probabilísticamente sobre propiedades de alcanzabilidad, donde se
dio una interpretación \emph{probabilística} a los operadores de selección. Más
específicamente, se pasa de una interpretación no determinística a una
probabilística y se estudia un sistema de tipos que permita determinar la
probabilidad con la que una sesión en particular termina \emph{exitosamente}.
Dado que no existe una interpretación universal de ``terminación exitosa'', se
diferencia la terminación exitosa de la infructuosa por medio de un constructor
dedicado.

Por ejemplo, podemos refinar \eqref{eq:bare.auction} como
\begin{equation}
    \label{eq:refined.auction}
    \cout\tint\parens{
        \tbranch[p]\tdone{
            \cin\tint\parens{
                \tchoice[q]\tend\T
            }
        }
    }
\end{equation}

donde el tipo sesión $\tdone$ indica una terminación exitosa y donde las
ramas  se anotan con probabilidades $p$ y $q$. En particular, el subastador
declara que el artículo se vende con probabilidad $p$ y  responde con una
contraoferta con probabilidad $1-p$, mientras que el comprador decide abandonar
la subasta con probabilidad $q$ o volver a ofertar con probabilidad $1-q$.
