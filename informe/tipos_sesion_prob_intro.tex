\label{cap:tipos_sesion_prob}
En~\cite{DBLP:conf/concur/InversoMPTT20} se propuso el uso de tipos de sesión
para razonar probabilísticamente sobre propiedades de alcanzabilidad, donde se
dio una interpretación \emph{probabilística} a los operadores de selección. Más
específicamente, se pasa de una interpretación no determinística a una
probabilística y se estudia un sistema de tipos que permita determinar la
probabilidad con la que una sesión en particular termina \emph{exitosamente}
(más precisiones sobre esta noción se darán en el Capítulo
\ref{cap:prob_exito}).  Dado que no existe una interpretación universal de
``terminación exitosa'', se diferencia la terminación exitosa de la infructuosa
por medio de un constructor dedicado.

Por ejemplo, podemos refinar \eqref{eq:bare.auction} como
\begin{equation}
    \label{eq:refined.auction}
    \T = \Out\tint{
        \BinaryPBranch[p]\Done{
            \In\tint{
                \BinaryPChoice[q]\Idle\T
            }
        }
    }
\end{equation}

donde el tipo sesión $\Done$ indica una terminación exitosa y las
ramas se anotan con probabilidades $p$ y $q$. En particular, el subastador
declara que el artículo se vende con probabilidad $p$ y responde con una
contraoferta con probabilidad $1-p$, mientras que el comprador decide abandonar
la subasta con probabilidad $q$ o volver a ofertar con probabilidad $1-q$.

\section{Gramática}
\label{sec:gramatica_prob}

Para esta interpretación probabilística consideramos la siguiente gramática

\[
\begin{array}{@{}r@{~~}c@{~~}l@{}}
\TypeT, \TypeS & ::= &
\tbool
\rulemid \tint
\rulemid \tvar
\rulemid \SessionType
\rulemid \ClosedSessionType
\rulemid \BinaryLabels{\SessionTypeT}{\SessionTypeS}
\rulemid \cdots
\\
\SessionTypeT, \SessionTypeS & ::= &
\Done
\rulemid \Idle
\rulemid \Out\Type\SessionType
\rulemid \In\Type\SessionType
\rulemid
\\
	& &
\BinaryPBranch[p]\SessionTypeT\SessionTypeS
\rulemid \BinaryPChoice[p]\SessionTypeT\SessionTypeS
\rulemid \etvarA
\rulemid \dual\etvarA
\end{array}
\]

donde $\TypeT$ y $\TypeS$ son utilizados para representar los  básicos,
variables libres, sesión para endpoints, sesiones con probabilidad de
éxito $p$ y otros. Las sesiones representadas por $\ClosedSessionType$ resultan
de la composición o ``unión'' de un par de endpoints con sesión duales
$\SessionTypeT$ y $\dual\SessionTypeT$ ambos con probabilidad de éxito $p$. La
\emph{probabilidad de éxito} de un tipo sesión es la probabilidad con la que
el protocolo descrito termina exitosamente.

A diferencia de la gramática descrita en la Sección \ref{sec:tipos_sesion_gram},
utilizamos dos constructores distintos para marcar el fin de una comunicación,
$\Idle$ y $\Done$. Utilizamos $\Done$ para denotar los puntos de terminación
exitosa de un protocolo y $\Idle$ para los de terminación infructuosa, la
semántica de terminación para ambos siendo dependiente del dominio del problema.

Los tipos sesión $\Out\Type\SessionType$ y $\In\Type\SessionType$ describen
endpoints para enviar y recibir mensajes al igual que en la Sección
\ref{sec:tipos_sesion_gram}. Luego, $\PChoice$ y $\PBranch$ describen ramas y
elecciones  ahora anotadas con una probabilidad y limitadas a una elección
binaria que es ``izquierda'' con probabilidad $p$ y ``derecha'' con
probabiliadad $1 - p$. El endpoint es luego consumido acorde $\SessionTypeT$ o
$\SessionTypeS$ según corresponda. La probabilidad de $\PBranch$ queda
determinada por su correspondiente elección interna. Por último, $\etvar$ y
$\dual\etvar$ representan variables libres y su dual.

El \emph{dual} de un tipo sesión probabilístico $\SessionType$, escrito como
$\dual\SessionType$, extiende la definición presentada en la Sección
\ref{sec:tipos_sesion_gram}; para los constructores de terminación es la
identidad y las operaciones de lectura y escritura son intercambiadas. Este
intercambio también se realiza para las elecciones internas y externas sin
afectar las probabilidades anotadas.

Por último, es necesario modificar la condición de alcanzabilidad para
considerar los nuevos constructores de terminación:
\begin{description}
	\item[Alcanzabilidad] Requerimos que todo sub-árbol $\SessionType$ de un
		tipo sesión contenga una \emph{hoja alcanzable} etiquetada por $\Done$ o
		$\Idle$.
\end{description}
