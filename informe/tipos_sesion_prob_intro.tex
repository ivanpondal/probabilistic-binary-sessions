En~\cite{DBLP:conf/concur/InversoMPTT20} se propuso el uso de tipos de sesión
para razonar probabilísticamente sobre propiedades de alcanzabilidad, donde se
dio una interpretación \emph{probabilística} a los operadores de selección. Más
específicamente, se pasa de una interpretación no determinística a una
probabilística y se estudia un sistema de tipos que permita determinar la
probabilidad con la que una sesión en particular termina \emph{exitosamente}.
Dado que no existe una interpretación universal de ``terminación exitosa'', se
diferencia la terminación exitosa de la infructuosa por medio de un constructor
dedicado.

Por ejemplo, podemos refinar \eqref{eq:bare.auction} como
\begin{equation}
    \label{eq:refined.auction}
    \Out\tint{
        \BinaryPBranch[p]\Done{
            \In\tint{
                \BinaryPChoice[q]\Idle\T
            }
        }
    }
\end{equation}

donde el tipo sesión $\Done$ indica una terminación exitosa y las
ramas se anotan con probabilidades $p$ y $q$. En particular, el subastador
declara que el artículo se vende con probabilidad $p$ y responde con una
contraoferta con probabilidad $1-p$, mientras que el comprador decide abandonar
la subasta con probabilidad $q$ o volver a ofertar con probabilidad $1-q$.

\section{Gramática}

Para esta interpretación probabilística consideramos la siguiente gramática

\[
\begin{array}{@{}r@{~~}c@{~~}l@{}}
\TypeT, \TypeS & ::= &
\tbool
\rulemid \tint
\rulemid \tvar
\rulemid \SessionType
\rulemid \ClosedSessionType
\rulemid \BinaryLabels{\SessionTypeT}{\SessionTypeS}
\rulemid \cdots
\\
\SessionTypeT, \SessionTypeS & ::= &
\Done
\rulemid \Idle
\rulemid \Out\Type\SessionType
\rulemid \In\Type\SessionType
\rulemid
\\
	& &
\BinaryPBranch[p]\SessionTypeT\SessionTypeS
\rulemid \BinaryPChoice[p]\SessionTypeT\SessionTypeS
\rulemid \etvarA
\rulemid \dual\etvarA
\end{array}
\]

donde $\TypeT$ y $\TypeS$ son utilizados para representar los  básicos,
variables libres, sesión para endpoints, sesiones con probabilidad de
éxito $p$ y otros. Las sesiones representadas por $\ClosedSessionType$ resultan
de la composición o ``unión'' de un par de endpoints con sesión duales
$\SessionTypeT$ y $\dual\SessionTypeT$ ambos con probabilidad de éxito $p$. La
\emph{probabilidad de éxito} de un tipo sesión es la probabilidad con la que
el protocolo descrito termina exitosamente.

Para los tipos sesión descritos por $\SessionTypeT$ y $\SessionTypeS$, $\Idle$
y $\Done$ marcan el fin de una comunicación. Utilizamos $\Done$ para denotar
los puntos de terminación exitosa de un protocolo, la semántica para tal
terminación siendo dependiente del dominio del problema. Los tipos sesión
$\Out\Type\SessionType$ y $\In\Type\SessionType$ describen endpoints para
enviar y recibir mensajes de tipo $t$ y continuar de acuerdo con $T$. Luego,
$\PChoice$ y
$\PBranch$ describen endpoints utilizador para
enviar y recibir una elección binaria que es ``izquierda'' con probabilidad $p$
y ``derecha'' con probabiliadad $1 - p$. El endpoint es luego consumido acorde
$\SessionTypeT$ o $\SessionTypeS$ según corresponda. Notamos que $\PChoice{}{}$
es una elección interna, el proceso comportándose acorde este tipo selecciona
``izquierda'' o ``derecha'', mientras que $\PBranch$ es una elección
externa, el proceso ofrece al exterior ambas ramas. Como consecuencia, la
probabilidad de $\PBranch$ queda determinada por su correspondiente
elección interna. Por último, $\etvar$ y $\dual\etvar$ representan variables
libres y su dual.

El \emph{dual} de un tipo sesión probabilístico $\SessionType$, escrito como
$\dual\SessionType$, se obtiene utilizando la misma definición que para los
tipos sesión intercambiando las operaciones de lectura y escritura.

Esta gramática permite describir también tipos sesión infinitos. Para ello
pedimos se cumplan las siguientes condiciones:
\begin{description}

	\item[Regularidad] Requerimos que todo árbol consista de un número
		finito de subárboles \emph{distinguibles}. Esta condición
		asegura que los tipos sesión sean representables mediante la
		notación $\mu$\todo{cite{Pierce02}} o como soluciones a un
		conjunto finito de ecuaciones\todo{cite{Courcelle83}}.

	\item[Alcanzabilidad] Requerimos que todo sub-árbol $\SessionType$ de
		un tipo sesión contenga una \emph{hoja alcanzable} etiquetada
		por $\Done$ o $\Idle$. Esta condición asegura que siempre sea
		posible terminar una sesión independientemente de hace cuánto
		esté corriendo.
\end{description}
