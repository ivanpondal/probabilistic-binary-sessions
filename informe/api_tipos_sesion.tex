\section{Gramática}

Consideramos la siguiente gramática

\[
\begin{array}{@{}r@{~~}c@{~~}l@{}}
\TypeT, \TypeS & ::= &
\tbool
\rulemid \tint
\rulemid \tvar
\rulemid \SessionType
\rulemid \set[i\in I]{ \Tag_i : \Type_i }
\rulemid \cdots
\\
\SessionTypeT, \SessionTypeS & ::= &
\End
\rulemid \Out\Type\SessionType
\rulemid \In\Type\SessionType
\rulemid \Branch \set[i\in I]{\Tag_i : \SessionType_i}
\rulemid \Choice \set[i\in I]{\Tag_i : \SessionType_i}
\rulemid \etvar
\rulemid \dual\etvar
\end{array}
\]

donde $\TypeT$ y $\TypeS$ son utilizados para representar los tipos básicos,
variables libres, tipos sesión, unión disjunta y otros.
Los tipos sesión quedan descritos por $\SessionTypeT$ y $\SessionTypeS$ con su
constructor para marcar el cierre de un endpoint, operaciones de
lectura/escritura, ramas y elecciones así como variables libres y su dual.

El \emph{dual} de un tipo sesión $\SessionType$, escrito como
$\dual\SessionType$, se obtiene intercambiando las operaciones de lectura y
escritura. Lo definen las siguientes ecuaciones:

\[
\begin{array}{c}
  \begin{array}{@{}r@{~~}c@{~~}l@{}}
    \smash{\dual{\dual\etvar}} & = & \etvar \\
    \dual\End & = & \End \\
  \end{array}
  \qquad
  \begin{array}{@{}r@{~~}c@{~~}l@{}}
    \dual{(\In\Type\SessionType)} & = & \Out\Type\dual\SessionType \\
    \dual{(\Out\Type\SessionType)} & = & \In\Type\dual\SessionType \\
  \end{array}
  \qquad
  \begin{array}{@{}r@{~~}c@{~~}l@{}}
    \dual{\Branch \set[i\in I]{\Tag_i : \SessionType_i}}
    & = & \Choice \set[i\in I]{\Tag_i : \dual{\SessionType_i}} \\
    \dual{\Choice \set[i\in I]{\Tag_i : \SessionType_i}}
    & = & \Branch \set[i\in I]{\Tag_i : \dual{\SessionType_i}} \\
  \end{array}
\end{array}
\]

\section{Interfaz programática}

\begin{table}[htb]
    \begin{OCamlD}[frame=single]
        val create  : unit -> $\etvar$ * $\sdual\etvar$
        val close   : $\End$ -> unit
        val send    : $\tvar$ -> $\Out\tvar\etvar$ -> $\etvar$
        val receive : $\In\tvar\etvar$ -> $\tvar$ * $\etvar$
        val select  : ($\sdual{\etvar_k}$ -> $\set[i\in I]{\Tag_i : \sdual{\etvar_i}}$) -> $\Choice \set[i\in I]{\Tag_i : \etvar_i}$ -> $\etvar_k$
        val branch  : $\Branch \set[i\in I]{\Tag_i : \etvar_i}$ -> $\set[i\in I]{\Tag_i : \etvar_i}$
    \end{OCamlD}
    \caption{Interfaz programática para tipos sesión}
    \label{tab:api}
\end{table}

En [CITA] se presentó el desarrollo de FuSe, una implementación en OCaml de la interfaz
de tipos sesión definida en Tabla~\ref{tab:api}.

\begin{itemize}
	\item \OI{create}: crea una nueva sesión y devuelve un par de endpoints con
tipos sesión duales.
	\item \OI{close}: señaliza el fin de una sesión.
	\item \OI{send}: envía mensaje de tipo $\tvar$ sobre endpoint con tipo
		$\Out\tvar\etvar$, retorna endpoint con tipo $\etvar$ para
		denotar el avance en la comunicación.
	\item \OI{receive}: espera recibir mensaje de tipo $\tvar$ sobre
		endpoint con tipo $\In\tvar\etvar$, retorna un par con
		mensaje y endpoint de tipo $\etvar$.
\end{itemize}

Las primitivas \OI{branch} y \OI{select} permiten tratar con sesiones que
presentan caminos con distintas interacciones, cada camino siendo identificado
por una \emph{etiqueta} $\TagVar_i$.
		
\begin{itemize}
	\item \OI{select}: envía etiqueta $\TagVar_k$ con $k\in I$
		sobre endpoint con tipo $\Choice \set[i\in I]{\Tag_i :
		\etvar_i}$, retorna endpoint $\etvar_k$ con tipo
		asociado a la etiqueta seleccionada. En \OCaml, el
		etiquetado es representado mediante una función que
		\emph{inyecta} un endpoint (ejemplo, el tipo
		$\sdual{\etvar_k}$) a la unión disjunta $\set[i\in I]{\Tag_i :
		\sdual{\etvar_i}}$ donde $k\in I$.
	\item \OI{branch}: dualmente, espera recibir etiqueta del
		endpoint con tipo $\Branch \set[i\in I]{\Tag_i : \etvar_i}$,
		retorna endpoint inyectado a la unión disjunta.
\end{itemize}
