\section{Gramática}

Consideramos la siguiente gramática

\[
\begin{array}{@{}r@{~~}c@{~~}l@{}}
\TypeT, \TypeS & ::= &
\tbool
\rulemid \tint
\rulemid \tvar
\rulemid \SessionType
\rulemid \set[i\in I]{ \Tag_i : \Type_i }
\rulemid \cdots
\\
\SessionTypeT, \SessionTypeS & ::= &
\End
\rulemid \Out\Type\SessionType
\rulemid \In\Type\SessionType
\rulemid \Branch \set[i\in I]{\Tag_i : \SessionType_i}
\rulemid \Choice \set[i\in I]{\Tag_i : \SessionType_i}
\rulemid \etvar
\rulemid \dual\etvar
\end{array}
\]

donde $\TypeT$ y $\TypeS$ son utilizados para representar los tipos básicos,
variables libres, tipos sesión, unión disjunta y otros.
Los tipos sesión quedan descritos por $\SessionTypeT$ y $\SessionTypeS$ con su
constructor para marcar el cierre de un endpoint, operaciones de
lectura/escritura, ramas y elecciones así como variables libres y su dual.

El \emph{dual} de un tipo sesión $\SessionType$, escrito como
$\dual\SessionType$, se obtiene intercambiando las operaciones de lectura y
escritura. Lo definen las siguientes ecuaciones:

\[
\begin{array}{c}
  \begin{array}{@{}r@{~~}c@{~~}l@{}}
    \smash{\dual{\dual\etvar}} & = & \etvar \\
    \dual\End & = & \End \\
  \end{array}
  \qquad
  \begin{array}{@{}r@{~~}c@{~~}l@{}}
    \dual{(\In\Type\SessionType)} & = & \Out\Type\dual\SessionType \\
    \dual{(\Out\Type\SessionType)} & = & \In\Type\dual\SessionType \\
  \end{array}
  \qquad
  \begin{array}{@{}r@{~~}c@{~~}l@{}}
    \dual{\Branch \set[i\in I]{\Tag_i : \SessionType_i}}
    & = & \Choice \set[i\in I]{\Tag_i : \dual{\SessionType_i}} \\
    \dual{\Choice \set[i\in I]{\Tag_i : \SessionType_i}}
    & = & \Branch \set[i\in I]{\Tag_i : \dual{\SessionType_i}} \\
  \end{array}
\end{array}
\]

\begin{table}
    \begin{OCamlD}[frame=single]
        val create  : unit -> $\etvar$ * $\sdual\etvar$
        val close   : $\End$ -> unit
        val send    : $\tvar$ -> $\Out\tvar\etvar$ -> $\etvar$
        val receive : $\In\tvar\etvar$ -> $\tvar$ * $\etvar$
        val select  : ($\sdual{\etvar_k}$ -> $\set[i\in I]{\Tag_i : \sdual{\etvar_i}}$) -> $\Choice \set[i\in I]{\Tag_i : \etvar_i}$ -> $\etvar_k$
        val branch  : $\Branch \set[i\in I]{\Tag_i : \etvar_i}$ -> $\set[i\in I]{\Tag_i : \etvar_i}$
    \end{OCamlD}
    \caption{Interfaz programática para tipos sesión}
    \label{tab:api}
\end{table}

\section{Interfaz programática}
