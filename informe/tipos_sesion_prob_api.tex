\section{Gramática}

Para esta interpretación probabilística consideramos la siguiente gramática

\[
\begin{array}{@{}r@{~~}c@{~~}l@{}}
\TypeT, \TypeS & ::= &
\tbool
\rulemid \tint
\rulemid \tvar
\rulemid \SessionType
\rulemid \ClosedSessionType
\rulemid \BinaryLabels{\SessionTypeT}{\SessionTypeS}
\rulemid \cdots
\\
\SessionTypeT, \SessionTypeS & ::= &
\Done
\rulemid \Idle
\rulemid \Out\Type\SessionType
\rulemid \In\Type\SessionType
\rulemid
\\
	& &
\BinaryPBranch[p]\SessionTypeT\SessionTypeS
\rulemid \BinaryPChoice[p]\SessionTypeT\SessionTypeS
\rulemid \etvar
\rulemid \dual\etvar
\end{array}
\]

donde $\TypeT$ y $\TypeS$ son utilizados para representar los  básicos,
variables libres,  sesión para endpoints, sesiones con probabilidad de
éxito $p$ y otros. Las sesiones representadas por $\ClosedSessionType$ resultan
de la composición o ``unión'' de un par de endpoints con  sesión duales
$\SessionTypeT$ y $\dual\SessionTypeT$.

Para los  sesión descritos por $\SessionTypeT$ y $\SessionTypeS$, $\Idle$
y $\Done$ marcan el fin de una comunicación. Utilizamos $\Done$ para denotar
los puntos de terminación exitosa de un protocolo, la semántica para tal
terminación siendo dependiente del dominio del problema. Los  sesión
$\Out\Type\SessionType$ y $\In\Type\SessionType$ describen endpoints para
enviar y recibir mensajes de tipo $t$ y continuar de acuerdo con $T$. Luego,
$\PChoice$ y
$\PBranch$ describen endpoints utilizador para
enviar y recibir una elección binaria que es ``izquierda'' con probabilidad $p$
y ``derecha'' con probabiliadad $1 - p$. El endpoint es luego consumido acorde
$\SessionTypeT$ o $\SessionTypeS$ según corresponda. Notamos que $\PChoice{}{}$
es una elección interna, el proceso comportándose acorde este tipo selecciona
``izquierda'' o ``derecha'', mientras que $\PBranch$ es una elección
externa, el proceso ofrece al exterior ambas ramas. Como consecuencia, la
probabilidad de $\PBranch$ queda determinada por su correspondiente
elección interna. Por último, $\etvar$ y $\dual\etvar$ representan variables
libres y su dual.

El \emph{dual} de un tipo sesión probabilístico $\SessionType$, escrito como
$\dual\SessionType$, se obtiene utilizando la misma definición que para los
 sesión intercambiando las operaciones de lectura y escritura.

\section{Interfaz programática}

Para la extensión a tipos sesión probabilísticos de FuSe se hicieron algunas
modificaciones a la interfaz presentada en Tabla~\ref{tab:api}.

\begin{table}[htb]
    \begin{OCamlD}[frame=single]
        val close        : $\Done$ -> unit
        val idle         : $\Idle$ -> unit
        val select_true  : $\BinaryPChoice[1]{\SessionTypeT}{\SessionTypeS}$ -> $\SessionTypeT$
        val select_false : $\BinaryPChoice[0]{\SessionTypeT}{\SessionTypeS}$ -> $\SessionTypeS$
        val pick         : $p$ -> $(\BinaryPChoice[0]{\SessionTypeT}{\SessionTypeS}$ -> $\alpha)$
                             -> $(\BinaryPChoice[1]{\SessionTypeT}{\SessionTypeS}$ -> $\alpha)$
                             -> $\BinaryPChoice[p]{\SessionTypeT}{\SessionTypeS}$ -> $\alpha$
        val branch       : $\BinaryPBranch[p]{\SessionTypeT}{\SessionTypeS}$
                             -> $\set{\Tag[True]: \SessionTypeT,\ \Tag[False]: \SessionTypeS}$
    \end{OCamlD}
    \caption{Interfaz programática para  sesión probabilísticos}
    \label{tab:prob_api}
\end{table}

Comenzamos con los cambios necesarios para la distinción de una terminación
exitosa, donde \OI{close} adquiere la semántica de terminación exitosa mientras
que \OI{idle} es utilizada para denotar el fin de la comunicación.

Procedemos con las modificaciones y nuevas primitivas para la selección de
caminos con cierta probabilidad. Las primitivas \OI{select_true} y
\OI{select_false} permiten avanzar la comunicación sobre un endpoint que
presenta una elección determinística por $\Tag[True]$ o $\Tag[False]$
respectivamente.

La función \OI{pick} (dada una probabilidad $p$) escoge $\Tag[True]$ con
probabilidad $p$ y $\Tag[False]$ con $p - 1$ sobre el endpoint con tipo
$\BinaryPChoice[p]{\SessionTypeT}{\SessionTypeS}$. El comportamiento sobre la
rama seleccionada queda definido por la función
$\BinaryPChoice[1]{\SessionTypeT}{\SessionTypeS} \to \alpha$ para el caso
$\Tag[True]$ y $\BinaryPChoice[0]{\SessionTypeT}{\SessionTypeS} \to \alpha$
cuando es $\Tag[False]$, ambas proceden con una elección determinística sobre
la rama escogida.

Por último \OI{branch} opera sobre una elección externa
$\BinaryPBranch[p]{\SessionTypeT}{\SessionTypeS}$ y retorna ambos caminos como
unión disjunta.
