\section{Gramática}

Para esta interpretación probabilística consideramos la siguiente gramática

\[
\begin{array}{@{}r@{~~}c@{~~}l@{}}
\TypeT, \TypeS & ::= &
\tbool
\rulemid \tint
\rulemid \tvar
\rulemid \SessionType
\rulemid \ClosedSessionType
\rulemid \cdots
\\
\SessionTypeT, \SessionTypeS & ::= &
\tend
\rulemid \tdone
\rulemid \cout\Type\SessionType
\rulemid \cin\Type\SessionType
\rulemid \tbranch[p]\SessionTypeT\SessionTypeS
\rulemid \tchoice[p]\SessionTypeT\SessionTypeS
\rulemid \etvar
\rulemid \dual\etvar
\end{array}
\]

donde $\TypeT$ y $\TypeS$ son utilizados para representar los tipos básicos,
variables libres, tipos sesión para endpoints, sesiones con probabilidad de
éxito $p$ y otros. Las sesiones representadas por $\ClosedSessionType$ resultan
de la composición o ``unión'' de un par de endpoints con tipos sesión duales
$\SessionTypeT$ y $\dual\SessionTypeT$.

Para los tipos sesión descritos por $\SessionTypeT$ y $\SessionTypeS$, $\tend$
y $\tdone$ marcan el fin de una comunicación. Utilizamos $\tdone$ para denotar
los puntos de terminación exitosa de un protocolo, la semántica para tal
terminación siendo dependiente del dominio del problema. Los tipos sesión
$\cout\Type\SessionType$ y $\cin\Type\SessionType$ describen endpoints para
enviar y recibir mensajes de tipo $t$ y continuar de acuerdo con $T$. Luego,
$\tchoice[p]\SessionTypeT\SessionTypeS$ y
$\tbranch[p]\SessionTypeT\SessionTypeS$ describen endpoints utilizador para
enviar y recibir una elección binaria que es ``izquierda'' con probabilidad $p$
y ``derecha'' con probabiliadad $1 - p$. El endpoint es luego consumido acorde
$\SessionTypeT$ o $\SessionTypeS$ según corresponda. Notamos que $\tchoice{}{}$
es una elección interna, el proceso comportándose acorde este tipo selecciona
``izquierda'' o ``derecha'', mientras que $\tbranch{}{}$ es una elección
externa, el proceso ofrece al exterior ambas ramas. Como consecuencia, la
probabilidad de $\tbranch{}{}$ queda determinada por su correspondiente
elección interna. Por último, $\etvar$ y $\dual\etvar$ representan variables
libres y su dual.
