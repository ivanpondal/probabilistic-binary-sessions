\section{Gramática}

Para esta interpretación probabilística consideramos la siguiente gramática

\[
\begin{array}{@{}r@{~~}c@{~~}l@{}}
\TypeT, \TypeS & ::= &
\tbool
\rulemid \tint
\rulemid \tvar
\rulemid \SessionType
\rulemid \ClosedSessionType
\rulemid \BinaryLabels{\SessionTypeT}{\SessionTypeS}
\rulemid \cdots
\\
\SessionTypeT, \SessionTypeS & ::= &
\Done
\rulemid \Idle
\rulemid \Out\Type\SessionType
\rulemid \In\Type\SessionType
\rulemid
\\
	& &
\BinaryPBranch[p]\SessionTypeT\SessionTypeS
\rulemid \BinaryPChoice[p]\SessionTypeT\SessionTypeS
\rulemid \etvarA
\rulemid \dual\etvarA
\end{array}
\]

donde $\TypeT$ y $\TypeS$ son utilizados para representar los  básicos,
variables libres,  sesión para endpoints, sesiones con probabilidad de
éxito $p$ y otros. Las sesiones representadas por $\ClosedSessionType$ resultan
de la composición o ``unión'' de un par de endpoints con  sesión duales
$\SessionTypeT$ y $\dual\SessionTypeT$.

Para los  sesión descritos por $\SessionTypeT$ y $\SessionTypeS$, $\Idle$
y $\Done$ marcan el fin de una comunicación. Utilizamos $\Done$ para denotar
los puntos de terminación exitosa de un protocolo, la semántica para tal
terminación siendo dependiente del dominio del problema. Los  sesión
$\Out\Type\SessionType$ y $\In\Type\SessionType$ describen endpoints para
enviar y recibir mensajes de tipo $t$ y continuar de acuerdo con $T$. Luego,
$\PChoice$ y
$\PBranch$ describen endpoints utilizador para
enviar y recibir una elección binaria que es ``izquierda'' con probabilidad $p$
y ``derecha'' con probabiliadad $1 - p$. El endpoint es luego consumido acorde
$\SessionTypeT$ o $\SessionTypeS$ según corresponda. Notamos que $\PChoice{}{}$
es una elección interna, el proceso comportándose acorde este tipo selecciona
``izquierda'' o ``derecha'', mientras que $\PBranch$ es una elección
externa, el proceso ofrece al exterior ambas ramas. Como consecuencia, la
probabilidad de $\PBranch$ queda determinada por su correspondiente
elección interna. Por último, $\etvar$ y $\dual\etvar$ representan variables
libres y su dual.

El \emph{dual} de un tipo sesión probabilístico $\SessionType$, escrito como
$\dual\SessionType$, se obtiene utilizando la misma definición que para los
 sesión intercambiando las operaciones de lectura y escritura.

\section{Interfaz programática}

Para la extensión a tipos sesión probabilísticos de FuSe se hicieron algunas
modificaciones a la interfaz presentada en Tabla~\ref{tab:api}.

\begin{table}[htb]
    \begin{OCamlD}[frame=single]
        val close        : $\Done$ -> unit
        val idle         : $\Idle$ -> unit
        val select_true  : $\BinaryPChoice[1]{\etvarA}{\etvarB}$ -> $\etvarA$
        val select_false : $\BinaryPChoice[0]{\etvarA}{\etvarB}$ -> $\etvarB$
        val pick         : $p$ -> $(\BinaryPChoice[0]{\etvarA}{\etvarB}$ -> $\alpha)$
                             -> $(\BinaryPChoice[1]{\etvarA}{\etvarB}$ -> $\alpha)$
                             -> $\BinaryPChoice[p]{\etvarA}{\etvarB}$ -> $\alpha$
        val branch       : $\BinaryPBranch[p]{\etvarA}{\etvarB}$
                             -> $\set{\Tag[True]: \etvarA,\ \Tag[False]: \etvarB}$
    \end{OCamlD}
    \caption{Interfaz para tipos sesión probabilísticos}
    \label{tab:prob_api}
\end{table}

Comenzamos con los cambios necesarios para la distinción de una terminación
exitosa, donde \OI{close} adquiere la semántica de terminación exitosa mientras
que \OI{idle} es utilizada para denotar el fin de la comunicación.

Procedemos con las modificaciones y nuevas primitivas para la selección de
caminos con cierta probabilidad. Las primitivas \OI{select_true} y
\OI{select_false} permiten avanzar la comunicación sobre un endpoint que
presenta una elección determinística por $\Tag[True]$ o $\Tag[False]$
respectivamente.

La función \OI{pick} (dada una probabilidad $p$) escoge $\Tag[True]$ con
probabilidad $p$ y $\Tag[False]$ con $p - 1$ sobre el endpoint con tipo
$\BinaryPChoice[p]{\etvarA}{\etvarB}$. El comportamiento sobre la
rama seleccionada queda definido por la función
$\BinaryPChoice[1]{\etvarA}{\etvarB} \to \alpha$ para el caso
$\Tag[True]$ y $\BinaryPChoice[0]{\etvarA}{\etvarB} \to \alpha$
cuando es $\Tag[False]$, ambas proceden con una elección determinística sobre
la rama escogida.

Por último \OI{branch} opera sobre una elección externa
$\BinaryPBranch[p]{\etvarA}{\etvarB}$ y retorna ambos caminos como
unión disjunta.

\subsection{Modificación al sistema de tipos de FuSe}

Los cambios presentados a la interfaz implican modificaciones al sistema de
tipos utilizado por FuSe. La sintaxis y semántica es esencialmente la estándar
para lenguajes de la familia ML sin ninguna extensión específica al chequeo de
tipos sesión.

Describiremos los cambios necesarios haciendo referencia al trabajo de FuSe
\todo{linkear paper de FuSe}. Las reglas \rulename{t-left} y \rulename{t-right}
utilizadas para tipar la expresión correspondiente al tipo unión disjunta
$\set{ \Tag[L] : T,\ \Tag[R] : S}$ se renombran a \rulename{t-true-branch} y
\rulename{t-false-branch} para coincidir con el tipo $\BinaryLabels{T}{S}$. Los
cambios restantes quedan restringidos a la regla \rulename{t-const}, que
describe los tipos de las primitivas de comunicación. En este caso las
modificaciones son análogas a las realizadas para la interfaz programática.
